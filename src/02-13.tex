\documentclass[10pt]{article}
\pdfinfoomitdate=1
\pdftrailerid{}
\usepackage{amsmath, amssymb}

\begin{document}

\section*{Math 505, 2/13}

We now introduce a new kind of variety, significantly different from an
affine variety. The motivation is that (despite the Nullstellensatz)
affine $n$-space $K^n$ is not large enough to solve problems involving
polynomials in a uniform way; we need to add certain points at infinity.
More precisely, start with the coordinate ring $S =
K[x_1,\ldots,x_{n+1}]$ of $K^{n+1}$ and observe (as mentioned last
quarter) that $S = \oplus_m S_m$ is a graded ring, where $S_m$ consists
of all homogeneous polynomials in $S$ of (total) degree $m$. If $p\in
S$, write $p_m$ for its homogeneous component of degree $m$, so that for
example $(x^2 y + x y^2 + xy)_3 = x^2 y + x y^2$. Call an ideal $I$ of
$S$ {\sl homogeneous} or {\sl graded} if it is the direct sum of its
projections $I_m$ to $S_m$ for $m\ge0$. The variety $V$ of such an ideal
will then be a {\sl cone}, that is, it will contain $kv$ for any $k\in
K$ whenever it contains $v)\in K^n$. We now define a new space $\mathbf
P^n$, called {\sl projective $n$-space}, which many of you will already
have seen in manifolds; it is obtained from $K^{n+1}$ by deleting the
origin and identifying $v\in K^{n+1},v\ne0$ with $kv$ for $k\in
K,k\ne0$. Given a homogeneous polynomial in $S$, it does not make sense
to evaluate it at an element $v$ of $\mathbf P^n$, but it does make
sense to ask whether or not it vanishes there. Accordingly, given a
radical homogeneous ideal $I\ne(x_1,\ldots,x_{n+1})$ of $S$, we define
its (projective) variety $V$ as the set of its common zeros in $P^n$,
relying on context to distinguish between varieties in $K^{n+1}$ and
those in $\mathbf P^n$; we have to omit the ideal $(x_1,\ldots,x_{n+1})$
since its only common zero is the origin, which has been deleted from
$\mathbf P^n$. (This ideal is therefore sometimes called the irrelevant
ideal.) Then the Nullstellensatz and a simple Vandermonde determinant
argument establish the {\sl homogeneous Nullstellensatz}, which states
the map $I\rightarrow V(I)$ sets up a 1-1 inclusion-reversing
correspondence between relevant radical homogeneous ideals in $S$ and
projective varieties; we also extend the Zariski topology to $\mathbf
P^n$ by decreeing that the closed sets are exactly the projective
varieties $V(I)$ of relevant homogeneous ideals $I$. (Note that the
radical of a homogeneous ideal is homogeneous, and that a homogeneous
ideal $I$ is prime if and only if it contains one of the factors $f$ or
$g$ of a product $fg$ of homogeneous polynomials whenever it contains
this product). Every projective variety $V=V(I)$ has a {\sl homogeneous
  coordinate ring} $S/I$ attached to it, but now the elements of $S/I$
are not well-defined functions on $V$. Nevertheless the ratio $f/g$ of
nonzero homogeneous polynomials in $S/I$ of the same degree {\sl is}
well defined as a function on $V$; the $K$-algebra consisting of all
such ratios is called the {\sl rational function field} $K(V)$ of $V$,
if $V$ is irreducible in the usual sense that $I$ is prime. (If $V$ is
irreducible and affine rather than projective, then we use the same
notation $K(V)$ and the same name \lq\lq rational function field" for
the full quotient field of the coordinate ring $K[v]$.)

Now a point $(x_1,\ldots,x_{n+1})$ in $\mathbf P^n$ with $x_1\ne0$ is
equivalent to a unique point $(1,y_1,\ldots,y_n)$ for some
$(y_1,\ldots,y_n)\in K^n$; note that $(y_1,\ldots,y_n)$ can even be the
origin. Hence there is a (principal) open subset $U_1$ of $\mathbf P^n$
and a natural bijection $\phi_1$ between it and affine $n$-space $K^n$;
even before we have defined the notion of morphism on a projective
variety in general, we decree that $\phi_1$ is an isomorphism of
varieties and the subring $K[x_2/x_1,\ldots,x_{n+1}/x_1]$ of the
quotient field of $S$ is the coordinate ring $K[U_1]$ of $U_1$. In a
similar manner, we define the affine open subset $U_i$ of $\mathbf P^n$
as the set of nonzeros of the coordinate function $x_i$ for $1\le i\le
n+1$; its coordinate ring $K[U_i]$ is the subring
$K[x_1/x_i,\ldots,x_{n+1}/x_i]$ of the quotient field of $S$. Then a
morphism between projective subvarieties $V,W$ of $\mathbf P^n,\mathbf
P^m$, respectively, is a Zariski-continuous map $\pi$ from $V$ to $W$
such that the restriction of $\pi$ to each affine open subset $V\cap
U_i$ of $V$ maps it into the affine open subset $W\cap U_j$ of $W$ for
some $j$ and is a morphism of affine varieties when so restricted. Thus
such morphisms, unlike those between affine varieties, need not be given
by a uniform formula (valid over all of $V$). The simplest example is
the subvariety $V$ of $\mathbf P^2$ defined by the homogeneous equation
$xz - y^2 = 0$. A point on this variety takes the form $(a^2,ab,b^2)$
with $a,b\in K^2, a,b$ not both 0, and $a$ and $b$ unique up to
multiplication by the same nonzero scalar. We can map this point to
$(a^2,ab)\in\mathbf P^1$ if $a\ne0$ and to $(ab,b^2)\in\mathbf P^1$ if
$b\ne0$; note that these formulas agree in $\mathbf P^1$ whenever both
are defined. This map is then an isomorphism from $V$ to $\mathbf P^1$;
its inverse sends $(a,b)\in\mathbf P^1$ to $(a^2,ab,b^2)$. Note also
that the homogeneous coordinate rings $K[x,y]$ and $K[x^2,xy,y^2]\subset
K[x,y]$ of $\mathbf P^1$ and $V$ are {\sl not} isomorphic, although
their rational function fields ({\sl not} the same as their quotient
fields) are isomorphic.

Thus $\mathbf P^n$ can be thought of as the overlapping union of $n+1$
copies of $K^n$. It can also be thought of as the disjoint union of
copies of $K^n,K^{n-1},\ldots,K^0$, by letting $Z_i$ be the complement
of $U_i$ and then taking the affine varieties $U_1,Z_1\cap U_2, Z_1\cap
Z_2\cap U_3$, and so on. Thus the copies of $K^{n-1},\ldots,K^0$ consist
exactly of the points \lq\lq at infinity" that one must add to $K^n$ to
get $\mathbf P^n$.

\end{document}
