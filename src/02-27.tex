\documentclass[10pt]{article}
\usepackage{amsmath, amssymb}

\begin{document}

\section*{Math 505, 2/27}

We are now ready to prove the results mentioned earlier about dimensions
of intersections of varieties in affine and projective space, but first
we need to discuss products of varieties. Given affine varieties $V,W$
lying in $K^n,K^m$, respectively, with corresponding ideals $I,J$ lying
in $K[x_1,\ldots,x_n],K[y_1,\ldots,y_m]$, we make the Cartesian product
$V\times W$ into a variety in a fairly obvious way, its ideal generated
by the copies of $I$ and $J$ inside the larger polynomial ring
$K[x_1,\ldots,x_n,y_1,\ldots,y_m]$. We insert a warning here that the
Zariski topology on $V\times W$ is {\sl not} the product of these
topologies on $V$ and $W$; it has many more closed sets. Nevertheless,
$V\times W$ is irreducible whenever $V,W$ are, for given a decomposition
$V_1\cup V_2$ of $V\times W$ into proper closed subsets, irreducibility
forces for each $w\in W$ the subset $S_w = \{v\in V: (v,w)\in V_1\}$ to
be either empty or all of $V$; then the union of all $w\in W$ for which
$S_w = V$ is either empty or all of $W$, whence finally $V_1 = V\times
W$ or $V_2 = V\times W$, as desired. The dimension of $V\times W$ is the
sum of the dimensions of $V$ and $W$, as one easily sees by constructing
an appropriate decreasing chain of varieties from $V\times W$ down to a
poin. We can similarly define the product of two projective varieties,
but this is trickier, as the product of subvarieties of $\mathbf
P^n,\mathbf P^m$ will not embed in $\mathbf P^{m+n}$ in any
straightforward way, thanks (or no thanks) to the equivalence relation
on coordinates in $\mathbf P^n$. We first define the product $\mathbf
P^n\times\mathbf P^m$ of $\mathbf P^n$ and $\mathbf P^m$ themselves;
this may be identified with the collection of all $nm+n+m+1$-tuples in
$\mathbf P^{mn+m+n}$ of the form
$(x_0y_0,\ldots,x_ny_0,x_0y_1,\ldots,x_ny_1,\ldots,x_ny_m)$; note that
these coordinates are replaced by equivalent ones if either or both of
the $x_i$ and $y_j$ are so replaced. The image of $\mathbf P^n\times
\mathbf P^m$ is then identified with the subvariety of $\mathbf
P^{nm+n+m}$ corresponding to the ideal which is the kernel of the map
$K[z_ij:0\le i\le n,0\le j\le m]\rightarrow K[x_0,\ldots x_n,y_0,\ldots
  y_m]$ sending $z_{ij}$ to $x_i y_j$ (as an exercise, write down
generators of this kernel). Then the product $V\times W$ of subvarieties
of $\mathbf P^n,\mathbf P^m$ is defined to be its image under the
embedding of $\mathbf P^n\times \mathbf P^m$ into $\mathbf P^{nm+n+m}$,
called the {\sl Segre embedding}. It too is irreducible if $V$ and $W$
are and has dimension the sum of the dimensions of $V$ and $W$. For
example, if $n=m=1$, then the image of this embedding is the subvariety
of $\mathbf P^3$ with defining equation $xy = zw$.

Now we can prove our dimension inequalities for intersections. Let $X,Y$
be irreducible affine subvarieties of $K^n$ of dimensions $r,s$,
respectively. First suppose that $Y$ is a hypersurface, defined by a
single equation $f=0$. Then we have seen in our treatment of dimension
theory that either $X\subset Y$ or every component of $X\cap Y$ has
dimension $r-1$, as desired. In general, look at the product $X\times
Y\in K^{2n}$. Let $\Delta = \{P\times P: P\in K^n\}$ be the diagonal
copy of $K^n$ in $K^{2n}$; the isomorphism $\Delta$ to $K^n$ identifies
$X\cap Y$ with $(X\times Y)\cap\Delta$. We intersect $X\times Y$ with
$\Delta$ by intersecting this product with each of the coordinate
hyperplanes $x_i - y_i = 0$ in turn; at each step, we either keep the
same dimension or drop in dimension by 1, so that in the end all
components of $X\cap Y$ have dimension at least $r+s-n$, as claimed (but
this intersection could be empty even if $r+s>n$). If instead $X,Y$ are
irreducible subvarieties of $P^n$, then we pass to their affine cones
$C(X),C(Y)$, of dimensions $r+1,s+1$ in $K^{n+1}$. If $r+s\ge n$, then
$r+1+s+1>n+1$, and $C(X),C(Y)$ must intersect in the origin at least, so
{\sl all irreducible components of $X\cap Y$ have dimension at least
  $r+s-n$ and this intersection is nonempty whenever $r+s\ge n$}.

Returning to dimension theory of Noetherian local rings, we wrap up this
topic with some final results. Let $A$ be Noetherian local with
dimension $d$ and maximal ideal $M$, We have seen that there are
$x_1,\ldots,x_d$ in $M$ such that the ideal $Q$ they generate contains a
power $M^k$ of $M$; we call the $x_i$ a {\sl system of parameters}. Let
$f(t_1,\ldots,t_d)$ be a homogeneous polynomial of degree $s$ with
coefficients in $A$ and assume that $f(x_1,\ldots,x_d)\in Q^{s+1}$. Then
we claim that {\sl all coefficients of $f$ lies in $M$}. Indeed, we have
seen that there is a surjection $\alpha$ from the polynomial ring over
$A/Q$ in indeterminates $t_1,\ldots,t_d$ to the graded ring $G$
corresponding to $A$ and the stable $Q$-filtration $(Q^n)$ of it,
sending $t_i$ to the image $\bar{x}_i$ of $x_i$ in $Q/Q^2\subset G$. The
hypothesis implies that $\bar{f}$, the reduction of $f$ mod $Q$, is in
the kernel of $\alpha$. If some coefficient of $f$ is not in $M$, so
that it is a unit, then $f$ is not a zero divisor, by a result in
upcoming homework; but then the dimension $d(G)$ of $G$ would then have
to be at most $d-1$, a contradiction. As a simple consequence, if there
is a subring $K$ of $A$ mapping isomorphically onto $A/M$ (as there is
for quotients of polynomial rings), then any set of elements
$x_1,\ldots,x_d$ as above is algebraically independent over $K$; thus
any quotient of $K[x_1,\ldots,x_n]$ of dimension $d$ contains a copy of
the polynomial ring $K[y_1,\ldots,y_d]$ (as we already knew thanks to
Noether normalization). Matters are even nicer (for general Noetherian
local rings) if the dimension of $M/M^2$ equals $d$, for then any
$K$-basis $y_1,\ldots,y_d$ of $M/M^2$, pulled back to elements
$x_1,\ldots,x_d$, is such that the quotient $M/I$ of $M$ by the ideal
$I$ generated by the $x_i$, equals $M(M/I)$, whence $M/I = 0, I = M$ by
Nakayama's Lemma. In this case the graded ring $G$ attached to the
standard $M$-filtration $(M^n)$ of $A$ is isomorphic to the polynomial
ring $K[y_1,\ldots,y_d]$. We call Noetherian local rings with this
property {\sl regular}; for example, the localization of any nonsingular
irreducible variety at any point is regular. In general the
$K$-dimension of $M/M^2$ is at least $d=\dim A$.

\end{document}
