\documentclass[10pt]{article}
\usepackage{amsmath, amssymb}

\begin{document}

\section*{Math 505, 2/15}

Continuing with both affine and projective varieties, we generalize the
notion of morphism to that of rational map: given irreducible varieties
$V,W$ a {\sl rational map} $\pi$ from $V$ to $W$ is a morphism from a
nonempty open subset $U$ of $V$ to $W$, with two such defined on open
subsets $U,U'$ being identified if they agree on the intersection $U\cap
U'$. Thus a rational map differs from a morphism in that it need not be
everywhere defined; in fact the restriction of any morphism to any open
subset of its domain is a rational map. Note that the notion of rational
map and the related notions below have no counterparts in manifold
theory; a crucial point here is that nonempty open subsets of
irreducible varieties, unlike those of manifolds, are always dense. A
rational map from $V$ to $W$ is called {\sl dominant} if its range is
dense in $W$; note that this property does not change if the map is
replaced by an equivalent one. Dominant rational maps from a variety $V$
to another one $W$ correspond bijectively to homomorphisms
$K(W)\rightarrow K(V)$ of their rational function fields. If two
rational maps $\pi:V\rightarrow W$ and $\phi:W\rightarrow V$ are such
that their composition in either order is identified with the identity
map, then we say that $V$ and $W$ are {\sl birationally equivalent} or
just {\sl birational}. The simplest example of a pair of nonisomorphic
birtaional varieties is an old friend from last week, namely the affine
variety $V$ defined by the equation $x^2 = y^3$ in $K^2$ and the affine
line $K^1$; here the respective coordinate rings are $K[x,y]/(x^2 -
y^3)\cong K[t^3,t^2]$ and $K[t]$. As you saw in homework last week,
these rings are not isomorphic and accordingly $V$ is not isomorphic to
$K^1$, but the rings become isomorphic after each is localized by powers
of $t$. Correspondingly we have a birational map from $K^1$ to $V$
defined on $K^*$ by sending $t$ to the pair $(t^3,t^2)$; its inverse is
defined on the complement $C$ of the origin in $V$ by sending $(x,y)$ to
$y/x$; this is a legitimate morphism on $C$ because its coordinate ring
picks up the function $x^{-1}$ that was not present in $K[V]$. In fact
$C$ is isomorphic as a variety to $K^*$, and indeed two varieties are
birational if and only if they have isomorphic open subsets, or
equivalently if and only if they ave isomorphic rational function
fields; in particular $K^n$ and $\mathbf P^n$ are birational. Assuming
for simplicity that the field $K$ has characteristic 0 (though the
following result holds in general), we know that the rational function
field $K(V)$ of any variety (affine, quasi-affine, projective, or
quasi-projective, the last denoting the intersection of a projective
variety and an open subset of projective space) is finitely generated as
a field extension of $K$, so that there are finitely many elements
$x_1\ldots,x_n$ of $K(V)$ that are algebraically independent over $K$
and $K(V)$ is a finite extension of the rational function field
$K_n=K(x_1,\ldots,x_n)$. By the Primitive Element Theorem from Galois
theory, there is a single element $y$ generating $K(V)$ as a
$K_n$-algebra; if $p$ is its minimal polynomial over $K_n$, then we can
clear denominators in $p$ to get a single irreducible polynomial $q$ in
$n+1$ variables over $K$ such that $q(x_1,\ldots,x_n,y) = 0$ and then
our variety $V$ is birational to the affine hypersurface in $K^{n+1}$
defined by the equation $q=0$.

We now want to extend a basic notion from manifold theory to algebraic
geometry, namely the definition of tangent space to a variety at a
point. Let $V\subset K^n$ be any affine variety, $v$ a point of $V$, and
$M_v$ the maximal ideal of the coordinate ring $K[V]$ corresponding to
the point $v$. Then the {\sl tangent space} $T_v(V)$ of $V$ at $v$ is
the dual of the $K$-vector space $M_v/M_v^2$; that is, it consists of
the $K$-linear maps from the quotient $M_v/M_v^2$ to $K\cong K[V]/M_v$.
(This definition parallels one you have seen or will see shortly in
manifolds.) Recall that a morphism $f$ from $V$ to another affine
variety $W$ induces a $K$-algebra homomorphism $f^*:K[W]\rightarrow
K[V]$ mapping $M_{f(v)}\subset K[W]$ for $v\in V$ to $M_v$, thus also
$M_{f(v)}^2$ to $M_v^2$. It follows that $f^*$ induces a $K$-linear map
called the {\sl differential} of $f$ and denoted by $df$ from $T_v(V)$
to $T_f(v) W$ (exactly as in manifolds). Now it turns out that we can
compute the dimension of the tangent space $T_v(V)$ of $V$ at $v$ in the
same way as we would for manifolds. Let $I$ be the ideal of $V$,
generated by the polynomials $f_1,\ldots,f_m$. Form the Jacobian matrix
$J_v$ whose $ij$-th entry is the partial derivative $\partial
x_i/\partial x_j$ evaluated at $v$ (computed via the usual formal rules;
we do not need limits). {\sl Then the corank of this matrix (that is,
  $n$ minus its rank), equals the dimension of $T_v(V)$; in particular,
  the corank of $J_v$ does not depend on the choice of generators of
  $I$. To see this, observe that the map sending any $f\in
  S=K[x_1,\ldots,x_n]$ to its gradient evaluated at $v$ vanishes on
  $M^2$ (by the product rule), where $M$ is the maximal ideal of $S$
  corresponding to $v$. It induces a vector space isomorphism between
  $M/M^2$ and $K^n$. Its value on a linear combination of the $f_i$ is a
  $K$-linear combination of the rows of $J_v$ (whose coefficients are
  those of the original linear combination evaluated at $v$). Hence the
  corank of $J_v$ matches the dimension of $T_v(V)$, as desired. Now
  this corank will generically (more precisely, on an open set, defined
  by the nonvanishing of one of a finite collection of determinants)
  take a certain value; we call such points {\sl regular} or {\sl
    nonsingular}. At other points, called {\sl singular} it will take a
  higher value. Thus {\sl the dimension of $T_v(V)$ takes a certain
    value on the set of regular points of $V$, a nonempty open set, and
    a higher value elsewhere}. It is natural to expect (and correct)
  that this value is none other than the dimension of $V$ itself
  (assuming $V$ is irreducible). We will prove this next time and then
  digress to discuss the notion of dimension in a more general
  context.For now, we mention a couple of examples. The variety
  $V\subset K^2$ with equation $x^2 - y^3 = 0$ has exactly one singular
  point, at $(0,0)$; there the tangent space is two-dimensional, while
  it is one-dimensional elsewhere. Similarly, the variety $W\subset K^3$
  defined by $xz - y^2 = 0$ has exactly one singular point, at the
  origin. The {\sl projective} variety defined by the same equation has
  no singular points, as $(0,0)$ is not a point in $\mathbf P^2$, and
  indeed we have observed that this last variety is isomorphic to
  $\mathbf P^1$.

\end{document}
