\documentclass[10pt]{article}
\pdfinfoomitdate=1
\pdftrailerid{}
\usepackage{amsmath, amssymb}

\begin{document}

\section*{Math 505, 2/22}

We continue to develop the theory of dimension, digressing for a moment
to introduce a measure of the rate of growth of the graded subspaces of
a graded module; we will then construct such a module from any
Noetherian local ring (thus also from any localization at a maximal
ideal of any Noetherian ring). So let $A=\oplus_0^\infty A_n$ be a
Noetherian graded ring with $A_0 = k$ a field (not assumed to be
algebraically closed), so that $A_n A_m\subset A_{n+m}$. Then the ideal
$A_+ = \oplus_1^\infty A_n$ is finitely generated by homogeneous
elements, say by $x_1\in A_{k_1},\ldots,x_s\in A_{k_s}$. An easy
induction then shows that the $n$th graded piece $A_n$ lies in the
$k$-algebra generated by the $x_i$ for all $n$, so that $A$ is a
finitely generated $k$-algebra. Let $M = \oplus_0^\infty M_n$ be a
finitely generated graded $A$-module, so that $A_n M_m\subset M_{n+m}$
and $M$ is generated by $m_1\in M_{r_1},\ldots,m_t\in M_{r_t}$. Then the
$n$th graded piece $M_n$ is spanned over $k$ by monomials of the
appropriate degree in the $x_i$ times generators $m_j$, whence $M_n$ is
finite-dimensional over $k$. Form the {\sl generating function} of the
sequence $\{\dim M_n\}$ of the dimensions of the $M_n$; that is, the
power series $P(M,t) = \sum_i \dim M_i t^i$; we call this the {\sl
  Hilbert series} or {\sl Poincar\'e series} of $M$. Then {\sl $P(M,t)$
  takes the form $f(t)/\prod_{i=1}^s (1-t^{k_i})$, for some $f\in\mathbf
  Z[t]$}. We prove this by induction on $s$. If $s=0$, then $A_n = 0$
for all $n>0$, whence $A= A_0 = k$ and $M$ is a finite-dimensional
vector space over $k$. In this case $M_n = 0$ for all large $n$ and
$f(t)$ is a polynomial, as desired. Now suppose that $s>0$ and the
theorem is true for $s-1$. Multiplication by $x_s$ gives an $A$-module
homomorphism from $M$ to itself sending $M_n$ to $M_{n+k_s}$, whence we
get an exact sequence
$$0\rightarrow K_n\rightarrow M_n\rightarrow M_{n+k_s}\rightarrow
L_{n+k_s}\rightarrow 0$$
\noindent for all $n$; the direct sums $K.L$ of the $K_n,L_n$,
respectively, are then finitely generated graded $A$-modules (being
respectively a submodule and a quotient of $M$) sent to 0 by $x_s$,
whence the induction hypothesis applies to them. Taking dimensions over
$k$ and using the additivity of dimension in exact sequences, we get
$\dim K_n - \dim M_n + \dim M_{n+k_s} - \dim L_{n+k_s} = 0$ for all
nonnegative $n$ and then $(1-t^{k_s})P(M,t) = P(L,t) - t^{k_s}P(K,t) +
g(t)$, where $g(t)$ is a polynomial over $\mathbf Z$ of degree at most
$k_s$. The inductive hypothesis then yields the desired result. The
order of the pole of $P(M,t)$ at $t=1$ is denoted $d(M)$; although this
quantity seems to be a million miles from chains of prime ideals, we
will define it for any Noetherian local ring and eventually relate it to
the Krull dimension of that ring. In case all $k_i$ happen to equal 1
(the main case of interest for us), we can refine this result: {\sl for
  sufficiently large $n$ the dimension $d_n$ of $M_n$ is a polynomial in
  $n$, called the Hilbert polynomial of $M$, of degree $d(M) - 1$}.
Indeed, we have $d_n =\,$coefficient of $t^n$ in $f(t)/(1-t)^s$ in this
case; cancelling a suitable power of $1-t$, we may assume that $s = d =
d(M)$ and $f(1)\ne0$. Write $f(t) = \sum_{k=0}^N a_k t^k$; then the
binomial theorem gives
$$(1-t)^{-d} = \sum_0^\infty {d+k-1\choose d-1} t^k$$
whence
$$ d_n = \sum_k a_k {d+n-k-1\choose d-1}$$
\noindent for $n\ge N$; the right side is a polynomial in $n$ of degree
$d-1$ and leading coefficient $(\sum a_k)/(d-1)!\ne0$, as desired. Thus
for example if $A=k[x_1,\ldots,x_n]$ has the standard grading, with $A_0
= k$, and $M=A$, then $d(M) = n$ (the Hilbert series of $M$ is
$(1-t)^{-n}$). Returning to the exact sequence above and replacing $x_s$
there by any $x\in A_k$ which is not a zero-divisor (in the sense that
$xm = 0$ implies $m=0$, for any $m\in M$, we see that $K=0$ and $d(L) =
d(M/xM) = d(M) - 1$.

We now show how to define $d(A), d(N)$ for any Noetherian local ring $A$
and any finitely generated $A$-module $N$. Let $M$ be the unique maximal
ideal of $A$. Form the {\sl associated graded ring} $G=G(A) =
\oplus_{i\ge0} G_i = \oplus_{i\ge0} (M^i/M^{i+1})$ in which addition is
defined componentwise and the product of $x\in G_i,y\in G_j$ is obtained
by taking the image $\bar{s}\bar{y}$ in $G_{i+j}$, where
$\bar{x},\bar(y)$ are any two preimages of $x,y$ in $M^i,M^j$,
respectively; one easily checks that this does not depend on the choice
of $\bar{x}$ or $\bar{y}$. Similarly define $G(N)$ to be the direct sum
of the quotients $G_{N,i} = M^i N/M^{i+1} N$, making this into a
$G$-module in the obvious way. Setting $K=A/M$, we see immediately that
$G$ is a $K$-algebra; if $m_1,\ldots,m_s$ generate $M$ as an ideal, then
the images of the $m_i$ in $G_1$ generate $G$ as a $K$-algebra, whence
$G$ is Noetherian (and in fact a quotient of a polynomial ring, such as
we have been working with in algebraic geometry). Similarly $G(N)$ is a
finitely generated graded $G$-module (generated by any set of generators
of $N$). Defining $d(G),d(G(N)$ as above, we then denote these
quantities by $d(A),d(N)$, respectively. Then {\sl the sum of the
  $K$-dimensions of the $G_i$ for $0\le i\le n$ is a polynomial $g(n)$
  for sufficiently large $n$ of degree $d(A)$}, since the difference
between the sum up to $n+1$ and the sum up to $n$ is a polynomial of
degree $d(A) - 1$ for sufficiently large $n$. Similarly the sum of the
$K$-dimensions of the $G_{N,i}$ for $0\le i\le n$ is a polynomial
$g_N(n)$ of degree $d(N)$ for sufficiently large $n$. More generally, we
could replace $M$ here by any ideal $Q$ lying between some power $M^k$
of $M$ and $M$, replacing $G(A)$ by $G_Q((A) = \oplus (q^i/Q^{i+1},
G(N)$ by $G_Q(N)$, defined similarly, and replacing the dimension of
$Q^n/Q^{n+1}$ over $K$ (which does not make sense) by the length of this
quotient as an $A$-module, which does make sense and equals the sum of
the $K$-dimensions of $Q^n/MQ^n, MQ^n/M^2Q^n,\ldots$, the sequence of
quotients stopping after at most $k$ steps since $M^k\subset Q$. The
leading coefficients of $g(n),g_N((n)$ then depend on the choice of $Q$,
but its degree $d(A)$ does not. Even more generally, we could replace
the multiples $Q^nN$ of $N$ by powers of $Q$ here by any {\sl stable
  $Q$-filtration} of $N$, that is, by any sequence $N_n$ of submodules
of $N$ such that $N_0 = N, QN_i\subset N_{i+1}\subset N_i$ for all $i$
and $N_{i+1}= QN_i$ for all sufficiently large $i$; then the polynomials
$g(n),g'(n)$ giving the lengths of $N/Q^{n+1}$ on the one hand and
$N/N_{n+1}$ on the other for sufficiently large $n$ are such that
$g(n)\le g'(n+n_0), g'(n)\le g'(n+n_0)$ for a fixed index $n_0$, so that
$g,g'$ have the same degree and leading coefficient.

We close here by recording two results that will be needed in homework
this week (which we are not quite ready to prove now): {\sl let $V,W$ be
  affine varieties in $K^n$ of respective dimensions $r,s$; then any
  component of $V\cap W$ has dimension at least $r+s-n$}. For projective
varieties $V,W$ in $\mathbf P^n$, we have a stronger result: {\sl under the
  same hypotheses, every component of $V\cap W$ has dimension at least
  $r+s-n$ and in addition $V\cap W$ is nonempty whenever $r+s-n\ge0$.}

\end{document}
