\documentclass[10pt]{article}
\pdfinfoomitdate=1
\pdftrailerid{}
\usepackage{amsmath, amssymb}

\begin{document}

\section*{Math 505, 1/30}

Having explored some of the properties of localization for general
commutative rings last time, we now look at how it works for integral
domains $D$. The first point is that the definition last time that two
fractions $a/s,b/t$ are equal if and only if there is $u$ in our
multiplicatively closed subset $S$ with $u(at - bs) = 0$ simplifies to
just $at - bs = 0$ (as in the construction of the full quotient field of
$D$), since $0\notin S$ and $D$ has no zero divisors. In particular, the
map from $D$ to $S^{-1}D$ sending $x$ to $x/1$ is 1-1 in this case. Now
specialize to the case where $D$ is a Dedekind domain. We have seen that
nonzero prime ideal $P$ of $D$ is maximal. The localization $D_P$ of $D$
at $P$, cutting out as it does all primes ideals of $D$ not contained in
$P$, leaves only two remaining prime ideals in $D_P$, namely 0 and
$Q=PD_P$ (the ideal of $D_P$ generated by $P$). But every nonzero ideal
of $D_P$, like every nonzero ideal of $D$, is a product of prime ideals,
so {\sl every nonzero ideal in $D_P$ is a power of $Q$}: the ideal
structure of $D_P$ is drastically simpler than it would be even for a
general PID. In fact, $D_P$ is a $PID$: we know that $Q\ne Q^2$ in
$D_P$, as in $D$, and if $x\in Q,x\notin Q^2$, then the principal ideal
$(x)$ is not contained in $Q^2$ or any higher power of $Q$, so it must
be all of $Q$. Thus {\sl every element of $D_P$ is a power of $x$ times
  a unit in $D_P$}. We call $D_P$ a {\sl discrete valuation ring}, or
$DVR$ for short; in fact we could have called it a $DVD$, since it is an
integral domain, but that abbreviation has been co-opted for another
purpose. We will give the reason for this terminology later. Perhaps the
simplest example is $\mathbf Z_{(p)}$ for $p$ a prime; this ring is not
the ring of integers mod $p$, but rather the the ring of all rational
numbers whose denominators are not divisible by $p$. A \lq\lq naturally
occurring" example (not arising by explicitly localizing another ring)
is the ring $K[[x]]$ of formal power series $\sum_{n=0}^\infty a_n x^n$
in one variable $x$ over a field $K$; here we impose no convergence
requirement on the power series. We add two power series $\sum a_n x^n,
\sum_n b_n x^n$ in the obvious way and multiply them via the rule
$\sum_n a_n x^n \sum_n b_n x^n = \sum_n c_n x^n$, where $c_m =
\sum_{n=0}^m a_n b_{m-n}$. At first you might think that the structure
of $K[[x]]$ would be more complicated than that of the polynomial ring
$K[x]$; in fact, it is much simpler, since an easy inductive argument
shows that any power series $\sum_{n=0}^\infty a_n x^n$ with $a_0\ne0$
is a unit in $K[[x]]$, so that any element of $K[[x]]$ is a power of $x$
times a unit, so that the only nonzero ideals of $K[[x]]$ are powers of
$(x)$. Returning now to a general discrete valuation ring $R$ whose
maximal ideal is generated by a single element $x$, we define a map $v$
from $R$ to the nonnegative integers by decreeing that $v(x^n u) = n$ if
$u$ is a unit in $R$, while $v(0)$ is undefined (or sometimes is taken
to be $-\infty$). Then we have $v(ab) = v(a) + v(b)$ for $a,b\ne0 in R$,
while $v(a+b)\ge\min(v(a),v(b))$ if $a,b,a+b\ne0$ in $R$. Such a map $v$
is called a {\sl discrete valuation} (discrete since its range lies in a
discrete set). We extend it to the quotient field $K$ of $R$ by
decreeing that $v(x^m u) = m$ for any integer $m$, positive or negative;
then we can recover $R$ from $K$ as the set of elements $x$ with
$v(x)\ge0$. In fact, given any field $K$ and a valuation $v$ from $K^*$
to $\mathbf Z$, the subring $R$ consisting of all $x\in K$ with
$v(x)\ge0$ is a discrete valuation ring; if $y\in R$ is such that
$v(y)=k>0$ and $k$ is minimal, then it is not difficult to check that
the principal ideal $(y)$ contains all $x\in K$ with $v(x)\ge k$ and
powers of this ideal account for all the nonzero ideals of $R$.
(Nondiscrete valuation rings, having valuations with ranges in other
ordered groups, can have a more complicated structure.) A famous example
of a discrete valuation ring, combining the features of the modular
integers and power series, is the ring of {\sl $p$-adic integers} for
$p$ a prime. As a set this is just $\mathbf Z_p[[x]]$, the power series
ring over the modular integers $\mathbf Z_p$, except that the variable
$x$ is replaced by $p$. The ring operations are those of $\mathbf
Z_p[[x]]$ with \lq\lq carrying", so that whenever a coefficient of a
power of $p$ exceeds $p$, we subtract off the appropriate multiple $kp$
of $p$ from it and then add $k$ to the coefficient of the next higher
power of $p$. Thus for example the sum of 1 and the series
$\sum_{n=0}^\infty (p-1)p^n$ is 0, so that this series equals $-1$, the
additive inverse of 1, while the product of $1+(p-1)p$ and the series
$\sum_{n=0}^\infty p^n$ is 1, so the series is the multiplicative
inverse of $1+(p-1)p$ in the $p$-adic integers. The quotient field
$\mathbf Q_p$ of the $p$-adic integers consists of all Laurent series in
$p$ (involving finitely many negative powers of $p$). It is usually
denoted $\mathbf Q_p$; similarly one often uses the notation $\mathbf
Z_p$ for the $p$-adic integers; but we will have no further occasion to
use them and so will reserve this notation for the integers mod $p$.

Before leaving Dedekind domains we mention one other result that will be
needed in this week's homework: given nonzero ideals $I,J$ in a Dedekind
domain $D$, there is an ideal $I'$ in the ideal class of $I$ that is
coprime to $J$ (so that $I'+J = D$). To prove this, begin as usual by
choosing $a\in I, a\ne0$ and write $IK = (a)$ for some ideal $K$ of $D$.
Then we have seen that $K$ is generated by $JK$ and one other element,
say $x$. Multiplying the equation $K = JK + (x)$ by $I$ and dividing by
$a$, we get $R = J + Ix/a$, whence $I' = Ix/a$ is an ideal coprime to
$J$ in the class of $I$, as desired. This result is needed to complete
the classification of finitely-generated torsion-free modules (and
ultimately all finitely generated modules) over $D$.



\end{document}
