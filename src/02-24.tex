\documentclass[10pt]{article}
\usepackage{amsmath, amssymb}

\begin{document}

\section*{Math 505, 2/24}

Continuing, we recall that $A$ is a Noetherian local ring with maximal
ideal $M$ and residue field $K=A/M$ and $N$ is a finitely generated
$A$-module. We saw last time that the lengths of the respective
quotients $A/M^n,N/M^nN$ as $A$-modules are polynomials $p_A,p_N$ of the
respective degrees $d(A),d(N)$ for sufficiently large $n$ and that these
degrees are unchanged if the power $M^n,M^nN$ are replaced respectively
by the ideal $Q^n,N_n$, where $Q$ lies between some power $M^k$ of $M$
and $M$ itself and the sequence $F=(N_n)$ is a stable $Q$-filtration of
$N$. We also saw that $d(A)$ is bounded above by $s$ whenever $M$ (or
$Q$) is generated by $s$ elements. Now let $N'$ be a submodule of $N$
and define a $Q$-filtration $(N_n')$ of $N'$ via $N_n'=N'\cap N_n$. Form
a new graded ring $B_Q(A) = \oplus_{i\ge0} Q^i$, called the {\sl blowup
  of $A$ at $Q$}, for which multiplication of $x\in Q^i,y\in Q^j$ is
defined as usual, putting the product in the $i+j$th graded piece
$Q^{i+j}$. Similarly define $B_F(N)$, the blowup of $N$ at $F=(N_n)$, to
be $\oplus_{i\ge0} N_i$; since $N_n$ is a $Q$-filtration, $B_F(N)$
becomes a graded $B_Q(A)$-module in a natural way. It is finitely
generated, by any set of generators of $N_0\oplus N_1oplus\ldots\oplus
N_i$, where $i$ is chosen large enough so that $N_{n+1} = QN_n$ for
$n\ge i$. Now we have an increasing chain of submodules $S_i$ of
$B_F(N')$ defined via $S_i =N_0' \oplus\ldots\oplus N_i'\oplus
QN_i'\oplus Q^2 N_i'\oplus\ldots$, which must terminate, forcing the
induced $Q$-filtration $(N_n')$ of $N'$ to be stable. Now let $x\in A,
x$ not a zero-divisor in $N$. Then $N' = xN\cong N$; setting $\bar{N} =
N/xN$ and $N_n'=N'\cap Q^nN$ as above, we have an exact sequence
$0\rightarrow N'/N_n'\rightarrow N/Q^n N\rightarrow \bar{N}/Q^n
\bar{N}\rightarrow 0$, whence $p_N'(n) - p_N(n) + p_{\bar{N}}(n) = 0$
for all sufficiently large $n$. Since $p_N',p_N$ have the same degree
and leading term, we see that $p_{\bar{N}}$ has degree at most $d(N) -
1$. In particular, {\sl $d(A/(x))\le d(A) - 1$ if $A$ is Noetherian
  local and $x$ is not a zero-divisor in $A$}. Now we can show that {\sl
  $\dim A\le d(A)$ for any Noetherian local ring $A$}. To prove this we
need a very useful result called {\sl Nakayama's Lemma: if $N$ is a
  finitely generated module over a local ring $A$ with $N=MN, M$ the
  maximal ideal of $A$, then $N = 0$}. Indeed, note first that any $x\in
A$ of the form $1+m$ for some $m\in M$ is a unit, since any non-unit
lies in a maximal ideal, but no such $x$ can lie in $M$. If $N$ is
finitely generated but $N\ne0$, let $n_1,\ldots,n_r$ be a minimal set of
generators; then we must have $n_1 = m_1 n_1 + \ldots + m_r n_r$ for
some $m_i\in M$, whence we can divide by $1-m_1$ and solve for $m_1$ in
terms of the remaining $m_i$, contradicting minimality of
$m_1,\ldots,m_r$ as a generating set for $N$. Next we prove $\dim A\le
d(A)$ by induction on $d(A)$. If $d(A)=0$, then the length of $A/M^n$ is
a constant for large enough $n$, forcing $M^n = M^{n+1}$ for large $n$;
since $M$ is finitely generated, Nakayama says that $M^n = 0$; but then
any prime ideal, containing 0, must contain $M$ and $M$ is the only
prime ideal, implying that $\dim A = 0$. If $d=d(A)>0$ and
$P_1\subset\cdots P_r$ is a strictly ascending chain of prime ideals in
$A$, then choose $x\in P_1,x\notin P_0$, and let $x'$ be the image of
$x$ in $A'=A/P_0$. Then $d(A'/(x')\le d(A') -1$; but the unique maximal
ideal $M'$ of $A'$ (the image of $M$) is such that $A'/(M')^n$ is a
homomorphic image of $A/M^n$ for all $n$, so that $d(A')\le d(A)$ and
$d(A'/(x'))\le d(A) - 1$. Then the images of $P_1,\ldots,P_r$ form a
strict chain of prime ideals in $A'$, whence $r\le d(A)$ and $\dim A\le
d(A)$, as desired. In particular, {\sl $\dim A$ is finite for any
  Noetherian local ring $A$}, Next, given any Noetherian local ring of
dimension $d$, we saw last time how to construct $x_1,\ldots,x_d\in A$
such that the only prime ideal of $A$ containing the $x_i$ is $M$ (the
unique prime ideal of height $d$), so {\sl $\dim A \ge \delta(A)$, where
  $\delta(A)$ is the minimum number of generators of any primary ideal
  with radical $M$}. Since the radical $r(I)$ of any ideal $I$ in $A$ is
finitely generated, say by $x_1,\ldots,x_m$ with $x_i^{n_i}\in I$, we
see that any product of $n_1+\ldots+ n_k$ combinations of the $x_i$ lies
in $I$, so $I$ contains a power of its radical. Hence {\sl the ideal
  $(x_1,\ldots,x_d)$ constructed above (where $d$ is the height of $M$),
  contains a power $M^k$ of $M$ and may be chosen as $Q$ in the recipes
  above for computing $d(A)$. We finally arrive at our central result:
  {\sl for any Noetherian local ring $A$, we have $d(A) = \dim A =
    \delta(A)$ and all three quantities are finite}. In particular, if
  $S=K[x_1,\ldots,x_n]$ with $K$ a field, then $\dim S = n$, since $d(S)
  = n$ (the Hilbert series of $S$ as a graded $S$-module is
  $(1-t)^{-n}$). If $A = S/P$ with $P$ a prime ideal and if $x\in
  A,x\ne0$, then $x$ is not a zero-divisor in $A$, whence {\sl any
    minimal prime ideal over $(x)$ has height 1}. If $K$ is
  algebraically closed, then this says that {\sl the intersection of an
    irreducible affine variety $V$ of dimension $d$ and a hypersurface
    $H$ not containing it has all irreducible components of dimension
    $d-1$}. In fact, {\sl given any irreducible varieties $V,W$ in $K^n$
    with $V\subset W$ and $\dim V\le\dim W - 2$, there is an irreducible
    variety $V'$ strictly between $V$ and $W$ and having dimension one
    less than that of $W$}: given the prime ideals $P_V,P_W$ of $V,W$,
  respectively, then $S/P_W$ has no zero divisors, so choose a nonzero
  $x$ lying in the image of $P_V$ in this quotient and look at a minimal
  prime ideal over $(x)$ contained in this image. Hence {\sl any two
    saturated chains of prime ideals in $S$, or of irreducible varieties
    in $K^n$, have the same length $n$}, as claimed previously; we also
  see for any irreducible variety $V$ that the dimension of the
  localization $K[V]_v$ of the coordinate ring $K[V]$ at any point $v\in
  V$ agrees with that of $V$. Given any irreducible projective variety
  $V\subset P^n$, defined by the homogeneous ideal $I\in S$, its affine
  cone $C(V)\subset K^{n+1}$ (defined by the same ideal) has dimension
  one larger, since we can compute the dimension of $V$ as the maximum
  length of any strict chain $I\subset\cdots\subset I_d$ of homogeneous
  prime ideals in $S$ whose last term $I_d$ is necessarily properly
  contained in the irrelevant ideal $S'=(x_1,\ldots,x_n)$, and then
  construct a corresponding chain to compute $\dim C(V)$ by adding $S'$
  to the first chain.

\end{document}
