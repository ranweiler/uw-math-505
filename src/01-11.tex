\documentclass[10pt]{article}
\usepackage{amsmath, amssymb}

\begin{document}

\section*{Math 505, 1/11}

We now look at a particularly important example of a Galois extension of
$\Bbb Q$, namely a {\sl cyclotomic extension} $C_n = \Bbb Q(e^{2\pi
  i/n})$ for some $n$. This is the splitting field of the polynomial
$x^n - 1$ over $\Bbb Q$, since the roots of $x^n - 1$ are clearly the
powers of $\alpha_n = e^{2\pi i/n}$. We begin by recalling that $x^n -
1$ is the product of the {\sl cyclotomic polynomials} $\Phi_d(x)$ as $d$
runs over the positive divisors of $n$, where $\Phi_d(x)$ the unique
monic polynomial whose roots are exactly the primitive $d$th roots of 1
in $\Bbb C$. It follows easily by (strong) induction on $n$ that
$\Phi_n(x)$ is a monic polynomial with integer coefficients; its degree
is $\phi(n)$, the number of positive integers less than $n$ and
relatively prime to it, or equivalently the order of the multiplicative
group $\Bbb Z_n^*$ of units in $\Bbb Z_n$. The main result is that {\sl
  $\Phi_n(x)$ is irreducible in $\Bbb Z[x]$ for any n}. To prove this
suppose for a contradiction that $\Phi_n(x)$ factors as $f(x) g(x)$
where $f(x),g(x)$ are monic with integer coefficients, $g(x)\ne1$, and
$f(x)$ is irreducible. Then every one of the $\phi(n)$ primitive $n$th
roots of 1 in $\Bbb C$ is a root of $f(x)$ or $g(x)$ but not both,
whence there is a prime number $p$ no dividing $n$ and a primitive $n$th
root $\alpha$ of 1 such that $\alpha$ is a root of $f(x)$ while
$\alpha^p$ is a root of $g(x)$. By irreducibility $f(x)$ must then
divide $g(x^p)$ in $\Bbb Z[x]$, since both of these polynomials have
$\alpha$ as a root; denoting by $\bar f(x),\bar g(x)$ the respective
reductions of $f(x),g(x)$ modulo $p$, we get that $\bar f(x)$ divides
$\bar g(x^p) = (\bar g(x))^p$ in $\Bbb Z_p[x]$. But then $x^n - 1$ would
have to have a repeated root in its splitting field over $\Bbb Z_p$
(since both $\bar f(x),\bar g(x)$ divide $x^n - 1$ in $\Bbb Z_p[x]$);
this is a contradiction, since the derivative $nx^{n-1}$ of $x^n - 1$
clearly has no roots in common with $x^n - 1$ over any field. Now we
know that all the powers $e^{2\pi i m/n}$ of $e^{2\pi i/n}$ are roots of
the same irreducible polynomial $\Phi_n(x)$ over $\Bbb Z$ or $\Bbb Q$,
as $m$ runs over the elements of $\Bbb Z_n^*$; it follows for any such
$m$ that there is a unique automorphism of $C_n$ sending $\alpha_n$ to
$\alpha_n^m$, so that the Galois group of $C_n$ over $\Bbb Q$ is exactly
$U_n = \Bbb Z_n^*$; in particular, it is abelian (and cyclic if $n$ is
prime, or a power of an odd prime).

It follows for any $n$ that any field $K$ between $\Bbb Q$ and $C_n$
that is Galois over $\Bbb Q$ has an abelian Galois group (being a
quotient of $U_n$). It is a remarkable fact that the converse holds:
{\sl any finite abelian extension of $\Bbb Q$, that is any finite Galois
  extension of $\Bbb Q$ with abelian Galois group, lies in $C_n$ for
  some $n$}. This result, called the Kronecker-Weber Theorem, at first
sight seems flatly impossible: if for example $p$ is a prime number, how
can the quadratic extension $\Bbb Q(\sqrt{p})$ of $\Bbb Q$, with Galois
group $\Bbb Z_2$, lie in any $C_n$? In fact, a fairly simple direct
calculation shows that it lies in $C_p$; extending this, it is not
difficult to show directly that any quadratic extension of $\Bbb Q$
indeed lies in a cyclotomic extension. Now suppose that $p$ is an odd
prime number of the form $2^m + 1$ for some $m$; it then turns out that
$m$ must itself be a power of 2, and in fact the there are only five
known examples, corresponding to the values $m=1,2,4,8,16$. Then the
cyclotomic extension $C_p$ has degree $\phi(p) = p-1 = 2^m$ over $\Bbb
Q$ and its Galois group is cyclic of this order. There is an obvious
descending chain of subgroups starting from $U_p$ and ending at 1, each
having index 2 in its predecessor; applying the Galois correspondence we
get an inreasing chain of fields starting at $\Bbb Q$ and ending at
$C_p$ with each a quadratic extension of its predecessor. The quadratic
formula then guarantees that each field can be obtained from its
predecessor by adjoining a single square root. But now there is a simple
geometric construction which starts from a line segment of a specified
length $a$ and constructs one of length $\sqrt{a}$ using only
straightedge and compass; in a similar manner, starting with a given
point $a+bi$ in the complex plane (together with the origin $0 = 0 +
0i$) and using only straightedge and compass, one can construct a second
point $c+di$ with $(c+di)^2 = a+bi$. The upshot is that {\sl the complex
  number $e^{2\pi i/p}$, or equivalently a regular $p$-gon inscribed in
  a unit circle, can be constructed using only compass and straightedge
  for any such $p$}. It was Gauss's discovery of this fact (for $p=17$)
that convinced him to go into mathematics as a profession. More recently
a rather anal-retentive German professor by the name of Hermes wrote a
manuscript for how this could be done explicitly for $p = 65537$, the
largest known prime of the form $2^m + 1$. This took ten years to
produce and the manuscript is carefully preserved under glass in
G\"ottingen today.

More generally (and more interestingly) one could ask for which $n$ is
there a formula for the roots of any polynomial over $\Bbb Q$ of degree
$n$ using only rational numbers, $m$-th roots (for any $m$, not just
$m\le n$), and field operations. We will see that such a formula exists
for $n=3$ or 4, but not any higher $n$; the proof will use the
simplicity of the alternating group $A_n$ for any $n\ge5$.

\end{document}
