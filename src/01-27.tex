\documentclass[10pt]{article}
\usepackage{amsmath, amssymb}

\begin{document}

\section*{Math 505, 1/27}

Now we can finally prove that the ring $R$ of algebraic integers in an
algebraic number field $L$ is a Dedekind domain. Given ideals $I,J$ with
$I\subset J$, we must show that $I=JK$ for some $K$. This trivially
holds if $J=R$; if it fails for any ideal $J$, there must be a largest
one for which it fails, so that it holds for any strictly larger ideal
$J'$. Choose $a\ne0$ in $J$ and let $P_1\cdots P_r$ be a minimal product
of nonzero prime ideals lying in the principal ideal $(a)$. Also choose
a maximal ideal $P$ containing $J$. Then $P$ contains the product
$P_1\cdots P_r$, whence it contains and must equal one of the $P_i$, say
$P_1$. Then the product $P_2\cdots P_r$ of fewer than $r$ prime ideals
cannot be contained in $(a)$, so there is $b\in R$ lying in this product
but not in $(a)$. Then $bJ\subset bP\subset bP_1\cdots P_r\subset (a)$,
whence $(b/a)J\subset R$, even though the fraction $b/a$ does not lie in
$R$ (since $b\notin (a)$). Now look at the ideal $J' = a^{-1}(a,b)J = J
+ (b/a)J$, where $(a,b)$ denotes the ideal generated by $a$ and $b$.
Since $b/a\notin R$ we deduce by a property of ideals of $R$ proved last
time that $(b/a)J$ is not contained in $J$, whence the ideal $J'$ is
strictly larger than $J$. By the choice of $J$ there is an ideal $K'$
with $I=J'K'$. Now set $K= a^{-1}(a,b)K'$. This is an $R$-submodule of
its quotient field $F$ and $JK = a^{-1}(a,b)JK' = J'K' = I$; since
$JK\subset J, K$ must be an ideal of $R$, not just a submodule of $F$.
This completes the proof.

We can measure precisely how close a Dedekind domain $D$ is to being a
PID via its {\sl class group}, the multiplicative group of (nonzero)
ideal classes. For a general Dedekind domain, this can be any abelian
group, but for the rings $R$ of algebraic integers in number fields $L$,
it turns out that this group is always finite. It is already very
interesting in the simplest special case $L=\mathbf Q(\sqrt{-m}$ for $m$
a positive nonsquare integer; here it turns out that the class group is
trivial exactly in the cases $m = 1,2,3,7,11,19,43,67,163$ and no
others! Gauss conjectured this result; it was finally proven in 1952,
with the proof not acknowledged to be correct until 1967. Thus the
simplest case where this ring is not a PID is the case $m=5$: here the
two essentially distinct factorizations $2\cdot 3 =
(1+\sqrt{-5})(1-\sqrt{-5})$ of 6 show that this ring admits nonprincipal
ideals.

We now show another way in which Dedekind domains are very close to
PIDs: given a nonzero proper ideal $I$ in such a domain $R$, the
quotient $R/I$ is such that every ideal is principal (though this
quotient generally has zero divisors). To see this let $M_1,\ldots M_k$
be the maximal ideals containing $I$, so that $I$ is the product of
powers of the $M_i$. Choose an element $x_1$ in $M_1$ but not in
$M_1^2$. The ideals $M_1^2,M_2,\ldots,M_k$ are pairwise coprime (i.e.
the sum of any two of them is $R$), so by the Chinese Remainder Theorem
there is $y_1\in R$ that is congruent to $x_1$ modulo $M_1^2$ and to 1
modulo any other $M_i$. Then the ideal generated by $I$ and $x_1$ lies
in $M_1$ but not in $M_1^2$ nor any other maximal ideal, so this ideal
must be $M_1$ itself. Hence the image $M_1/I$ in $R/I$ is principal,
being generated by $y_1$, and similarly the image $M_i/I$ is also
principal for all other $i$. Since any ideal $J$ in $R/I$ is a product
of maximal ideals $M_i/I$ it too must be principal. Thus every ideal in
any Dedekind domain is generated by at most two elements.

It now turns out that Dedekind domains are closely related to PIDs in
yet another way, one which motivates a general construction that is a
basic tool in commutative algebra. It is called {\sl localization}; we
have seen it briefly in connection with working out the behavior of
$\mathbf Z$-modules when tensored with $\mathbf Q$. Let $S$ be a
multiplicatively closed subset of a (general) commutative ring $A$, so
that $1\in S$ and $S$ is closed under multiplication; to avoid
trivialities we also assume that $0\notin S$. We now generalize the
construction of the quotient field of an integral domain to $A$, even
though $A$ may contain zero divisors. Let $S^{-1}A$, the {\sl
  localization of $A$ at $S$}, consist of all formal fractions $a/s$
with $a\in A, s\in S$. We decree that $a/s = b/t$ if and only if there
is $u\in S$ with $u(at - bs) = 0$; one easily checks that this is an
equivalence relation (but it would not be if the relation were just $at
- bs = 0$). We then add and multiply fractions in the standard way from
high-school algebra, checking that this is well-defined on equivalence
classes. There is a natural homomorphism from $A$ to $S^{-1}A$, sending
$a$ to $a/1$, which is {\sl not} in general 1-1; its kernel is the set
of all $a\in A$ such that $sa = 0$ for some $s\in S$. Then every ideal
$J$ of $S^{-1}A$ is generated by its intersection $I$ with $A$, which is
an ideal of $A$ (look at the numerators of elements of $J$); if $I$ has
nonempty intersection with $S$, then $J$ blows up in the sense that it
is all of $S^{-1}A$. The most important example occurs when $S$ is the
{\sl complement} in $A$ of a (proper) prime ideal $P$ of $A$ (so that
$S$ is multiplicatively closed by definition of a prime ideal). We
denote $S^{-1}A$ by $A_P$ in this case and call it (by abuse of
terminology) the localization of $A$ at $P$. Passing from $A$ to $A_P$
thus cuts out all ideals not contained in $P$; it is not difficult to
check conversely that {\sl there is a 1-1 correspondence between prime
  ideals of $A$ contained in $P$ and prime ideals of $A_P$ (in fact, for
  any multiplicatively closed subset $S$, there is a 1-1 correspondence
  between prime ideals of $A$ not meeting $S$ and prime ideals of
  $S^{-1}A$, mapping $I$ to $S^{-1}I$}. We will say more about
localization at prime ideals in the context of Dedekind domains next
time.

\end{document}
