\documentclass[10pt]{article}
\usepackage{amsmath, amssymb}

\begin{document}

\section*{Math 505, 2/8}

For the rest of the course let $K$ be an algebraically closed field.
Given varieties $V,W$, lying in $K^n,K^m$, respectively, we need to say
what mappings are allowed between $V$ and $W$. A natural choice is the
polynomial maps (sending $v\in V$ to $w=(f_1(v),\ldots,f_m(v)$ for some
$f_i\in K[x_1.\ldots,x_n]$); these are called {\sl morphisms}. Any
choice of the $f_i$ will define a morphism from $K^n$ to $K^m$, but in
order to define a map from $V$ to $W$, it must be the case that if $v$
is a common zero of the (radical) ideal $I$ corresponding to $V$, then
$w$ must be a common zero of the (radical) ideal $J$ corresponding to
$W$. We now observe that the $f_i$ also define a unique homomorphism
from the polynomial ring $K[y_1,\ldots,y_m]$ to $K[x_1,\ldots,x_n]$,
sending $y_i$ to $f_i$, and that every such homomorphism takes this form
for unique $f_1,\ldots,f_m$; it will induce a map from $V$ to $W$ if and
only if it takes the ideal $J$ in $K[y_1,\ldots,y_m]$ to $I$, or
equivalently it induces a well-defined map from $K[W] =
K[y_1,\ldots,h_m]/J$ to $K[V] = K[x_1,\ldots,x_n]/I$. We deduce that
{\sl there is a natural 1-1 correspondence between morphisms from $V$ to
  $W$ and algebra homomorphisms from $K[W]$ to $K[V]$} (so that the map
sending $V$ to its coordinate ring $K[V]$ is a contravariant functor, in
the language we used last quarter). In particular, our morphism from $V$
to $W$ is an isomorphism (i.e. has an inverse which is also a morphism)
if and only if the homomorphism from $K[W]$ to $K[V]$ is an algebra
isomorphism.

The Noether normalization theorem that we used to prove the weak Hilbert
Nullstellensatz furnishes some especially interesting examples of
morphisms. We have shown that the coordinate ring $K[V]$ of any variety
$V$ is a (finitely generated) integral extension of some polynomial ring
$K[x_1,\ldots,x_m]$ over $K$, so that there is a natural inclusion of
$K[x_1,\ldots,x_m]$ in $K[V]$; going backwards to the corresponding
morphism, we see that {\sl there is a surjective morphism $\pi$ from $V$
  to the affine space $K^m$}, which we will later see has finite fibers
(i.e. the inverse image $\pi^{-1}(v)$ of any $v\in K^m$ is finite. (We
will also see that the integer $m$ here is uniquely determined by $V$
and is naturally enough called its {\sl dimension}.) We call $\pi$ a
{\sl ramified finite cover}; it is analogous to the covering maps one
studies in topological manifolds, but is less well behaved and in
particular does {\sl not} define a local homeomorphism between any
neighborhoods in $V$ and $K^m$. For example, look at the variety $W$ in
$K^2$ consisting of the zeros of the single polynomial $x^2 - y^3$.
There are two obvious surjective morphisms from $W$ to the line $K^1$,
given by the projections $\pi_1,\pi_2$ onto the first and second
coordinates, respectively. The first map is generically a triple cover;
for any $x\ne 0$ there will be three distinct $y\in K$ with $x^2 = y^3$,
but for $x=0$ there is only one such $y$. Likewise, for any $y\ne0$
there are generically two distinct $x\in K$ with $x^2 = y^3$, but for
$y=0$ there is only one such $x$. It is because the fibers have
different sizes that we call such a cover ramified (=branched). There is
another very interesting algebra map, this time from $K[W]$ to $K[x]$,
which you will define in homework for this week. The corresponding
morphism is bijective but not an isomorphism, since its inverse is not a
morphism. As another example, look at the variety $V$, again in $K^2$,
defined by the equation $xy = 1$. This variety again admits projections
$\pi_1,\pi_2$ to the first and second coordinates, but this time the
$\pi_i$ are not surjective. Accordingly the corresponding algebra maps,
sending $K[x],K[y]$ respectively to their canonical images in $K[V]\cong
K[x,y]/(xy - 1)\cong K[x,x^{-1}]$, a localization of $K[x]$, does {\sl
  not} realize $K[V]$ as an integral extension of either image (though
$K[V]$ is in fact an integral extension of the polynomial ring $K[z]$
for a different embedding of $K[z]$ in $K[V]$, as you will work out in
another homework problem). We can use these maps to give the image $K^*$
of $\pi_1$ or $\pi_2$ the structure of an affine variety, as follows;
note that $K^*$ is not a closed subset of $K$ and thus not as it stands
affine. More generally, let $W$ be any affine variety in $K^n$ and $W'$
its intersection with a (Zariski-)open subset $U$ of $K^n$. Call an
element $f/g$ of the quotient field of $K[W]$ {\sl regular on $W'$} if
the denominator $g$ never vanishes at any point of $W'$; denote by
$K[W']$ the algebra of regular functions on $W'$. In practice we can
restrict to the case where $U$ is principal open; that is, it is the set
of nonzeros of a single polynomial $f$; then $K[W']$ is just the
localization $K[W]_f$ of $K[W]$ by powers of $f$. This is a finitely
generated $K$-algebra, isomorphic to the coordinate ring of the variety
$V = \{(x_1,\ldots,x_{n+1}\in K^{n+1}: (x_1,\ldots,x_n)\in W,
f(x_1,\ldots,x_n) x_{n+1} = 1\}$; accordingly we declare that $W'$ is an
affine variety isomorphic to $V$ with the morphism from $V'$ to $W'$
given by projection onto the first $n$ coordinates.

 Still more generally, we call any intersection $V\cap U$ of an affine
 variety with an open subset a {\sl quasi-affine} variety:; in general
 it is not isomorphic to any affine variety. We will look at another
 class of naturally occurring non-affine varieties later (the projective
 ones).

\end{document}
