\documentclass[10pt]{article}
\pdfinfoomitdate=1
\pdftrailerid{}
\usepackage{amsmath, amssymb}

\begin{document}

\section*{Math 505, 2/17}

We now show that the generic dimension of the tangent space of an
(irreducible) variety $V$ matches the dimension of $V$ itself. Replacing
$V$ by a suitable affine open subset and using that the dimension of a
typical quotient $M_v/M_v^2$ is unchanged if the coordinate ring $K[V]$
is replaced by its localization $K[V]_{M_v}$, we may assume by a result
proved last time that $V$ is a hypersurface in $K^n$, defined by a
single equation $f=0$ for some irreducible $f\in S=K[x_1,\ldots,x_n]$;
then the principal ideal $(f)$ generated by $f$ is clearly minimal, not
properly containing any other nonzero prime ideal, so the dimension of
$V$ is $n-1$. Now we just have to see that the set of points in $V$ at
which the gradient of $f$ vanishes, which is closed in $V$, is a proper
subset, so that the gradient of $f$ is not identically 0 on $V$. Since
$f$ is irreducible and its partial derivatives have lesser degree, the
gradient $\nabla f$ of $f$ vanishes on $V$ if and only if $\nabla f =
0$. This is clearly impossible in characteristic 0; in prime
characteristic $p$, it would force $f$ to be a polynomial in the $p$th
powers $x_1^p,\ldots,x_n^p$ of the variables $x_i$, whence by taking
$p$th roots of its coefficients (possible since $K$ is algebraically
closed) we would have $f = g^p$ for some $g\in S$, contradicting the
irreducibility of $f$. Thus the generic dimension of the tangent space
of $V$ indeed matches its dimension $n-1$, as desired, and at nongeneric
points the dimension of the tangent space increases. (Thus our point of
view is different than in Math 126, for example, where we would say
instead that the tangent space is undefined at any point where the
Jacobian matrix has the wrong corank. Note also that such points would
be excluded by definition in any smooth manifold.) For reducible
varieties $V$, we define the tangent space in the same way and the same
result holds, but note here that singular points can arise when
irreducible components intersect; for example, the origin is a singular
point of the variety with equation $xy = 0$, even though this point is
nonsingular in both the $x$- and $y$-axes, Finally, observe that (as
mentioned previously) the origin is singular in our old friend the curve
with equation $x^2 - y^3 = 0$, so that the coordinate ring of this curve
is \lq\lq bad" in another sense (in addition not to being integrally
closed), in that it is singular at the maximal ideal $(x,y)$. (It turns
out that these two failures, one of nonsingularity and the other of
integral closedness, are equivalent for curves, but not for
higher-dimensional varieties.)

We now discuss the behavior of dimension for general commutative rings
$A$. So far, we have used the Krull dimension, according to which the
dimension of $A$ is the largest $n$ for which there is a strictly
ascending chain $P_0\subset\cdots\subset P_n$ of prime ideals of $A$, or
$\infty$ if arbitrarily long such chains exist). Later we will give two
other definitions of dimension for Noetherian local rings (having
exactly one maximal ideal), which turn out to be equivalent to this one;
recall for any prime ideal $P$ of such a ring that $A_P$ is Noetherian
local (Noetherian because any ideal of its takes the form $I_P$ for some
ideal of $A$ and so is generated by any set of generators in $A$; local
because its unique maximal ideal is $PA_P$). For now we want to
generalize the notion of irreducibility (of varieties) to any
commutative Noetherian ring $A$. Call an ideal $I$ of $A$ {\sl
  irreducible} if it is not the intersection of two ideals properly
containing it. Then one easily checks that any ideal of $A$ is a finite
intersection of irreducible ones, which is unique up to reordering if
none of them contains another (the proof is parallel to our proof that
every variety in $K^n$ is uniquely a finite union of irreducible ones).
Now we observe that {\sl any irreducible ideal $I$ is {\bf primary} in
  the sense that if it contains a product $xy$, it contains either $x$
  or a power $y^n$ of $y$}. By passing to the quotient it is enough to
prove this for $I=0$; suppose accordingly that $xy=0$ but $y\ne0$. Let
$I_n = \{z\in A: x^nz = 0\}$; then the $I_n$ form an ascending chain of
ideals in $A$, which must stabilize, so that $I_n = I_{n+1}$ for some
$n$. Then $(x^n)\cap (y) = 0$, for if $a\in (y)$, then $ax = 0$; if also
$a\in (x^n)$, then $a=bx^n$ for some $b$, whence $ax = 0$ forces $b\in
I_{n+1} = I_n$, and $a=0$, as claimed. But then $(0)$ is the
intersection of $(x^n)$ and $(y)$, whence irreducibility forces $x^n =
0$, as desired. It follows easily that {\sl any irreducible ideal has
  prime radical} and thus that {\sl any radical ideal $J$ of $A$ is a
  finite intersection of prime ideals $P_i$, which is unique up to
  reordering if no $P_i$ contains another} (just like radical ideals in
polynomial rings). We call the $P_i$ the {\sl minimal primes over $J$};
note that these same prime ideals may be regarded the minimal ones over
any ideal with radical $J$. Given now a prime ideal $P$ of finite {\sl
  height} $m$ (so that there is a strictly ascending chain of prime
ideals $P_0\subset\cdots\subset P_m=P$, but no longer such chain),
choose elements $x_1,\ldots,x_m$ of $P$ such that, for all indices $i\le
m$, every prime ideal of $A$ containing $x_1,\ldots,x_i$ has height at
least $i$ inductively, as follows. This is trivial for $i=0$; if $i>0$
and $x_1,\ldots,x_{i-1}$ have been constructed, let $P_1\ldots,P_m$ be
the minimal primes of $(x_1,\ldots,x_{i-1}$ of height exactly $i-1$, if
any; if there are none, then we can just set $x_i = 0$. Then the ideal
$P$ cannot be contained in any of the $P_i$, since its height is larger
than $i-1$. We claim that $P$ cannot be contained in their union either.
Indeed, if we inductively choose $y_j$ in $P$ but not in any $P_k$ for
any index $k<i$ with the possible exception of $k=j$, then if any $y_j$
fails to lie in $P_j$, then we are done and can take $x_i = y_j$;
otherwise the sum of products $\sum_{j=1}^{i-1} \prod_{k\ne j}
(x_1\cdots x_k\cdots x_{i-1}$ lies in $P$ but not any $P_i$ and so can
be taken as $x_i$; in either case $x_1,\ldots,x_i$ have the desired
property. Thus $P$ is one of the minimal primes of $J=(x_1,\ldots,x_m)$
in $A$. If $A$ is local and $P$ is its unique maximal ideal, then the
radical of $J$ must be $P$, if $P$ has finite height $h$, since then $P$
is the only prime ideal of this height. We will give two other
characterizations of dimension and prove that the dimension of
$K[x_1,\ldots,x_n]$ is $n$ next week.

\end{document}
