\documentclass[10pt]{article}
\usepackage{amsmath, amssymb}

\begin{document}

\section*{Math 505, 3/1}

Now we get to become little kids again and blow things up. More
precisely, given a commutative Noetherian ring $A$ and an ideal $I$ of
$A$, we have defined $B_I(A)$, the blowup algebra of $I$ in $A$, to be
the graded direct sum $\oplus_{i\ge0} I^i$, where $I^0$ is taken to be
$A$. The associated graded ring $G_I(A)$ identifies with
$B_I(A)/IB_I(A)$. In a similar way, if $M$ is a finitely generated
$A$-module equipped with an $I$-filtration $F=(M_i)$, then $B_F(M)$, the
{\sl blowup module of $F$ in $M$} is defined to be the graded
$B_I(A)$-module $\oplus_{i\ge0} M_i$. This module is finitely generated
if and only if the filtration $F$ is stable. The main case of interest
in algebraic geometry occurs when $A$ is the coordinate ring of an
affine variety $V$ and $I$ is the ideal (not necessarily maximal)
corresponding to a subvariety $W$. For example, take $V$ to be $K^n$ and
$W$ the origin, so that $A=S=K[x_1,\ldots,x_n], M= (x_1,\ldots,x_n)$.
Then the variety $\mathbf B$ corresponding to $B_M(S)$ may be described
via affine coordinates $x_1,\ldots,x_n$ and projective coordinates
$y_1,\ldots,y_n$ (so that the $y_i$ are not allowed to be simultaneously
0 and we may replace $y_1,\ldots,y_n$ by $ky_1,\ldots,ky_n$ for any
$k\in K^*$). The defining equations are then $x_i y_j = x_j y_i$ for all
$i\ne j$. We then have a projection map $\phi:\mathbf B\rightarrow K^n$
sending $(x_1,\ldots,x_n,y_1,\ldots y_n)$ to $(x_1,\ldots,x_n)$ whose
fiber over any point $v\ne0$ in $K^n$ is a single point in $\mathbf
P^{n-1}$, but whose fiber over 0 is all of $\mathbf P^{n-1}$. In effect,
we have replaced the origin in $K^n$ by a copy of $\mathbf P^{n-1}$
(this is why we call the operation \lq\lq blowing up"). More generally,
let $V$ be a subvariety of $K^n$ having the origin 0 as a singular
point. Then the closure of the inverse image $\phi^{-1}(V-0)$ in any of
the affine open pieces covering $\mathbf B$ (where $V-0$ denotes $V$
with 0 removed) is called the {\sl blowup} or {\sl strict transform} of
$V$. Revisiting once again our favorite example, the variety $V$ defined
by $x^2 = y^3$, we find that a dense subset of its blowup is defined by
the two equations $x^2 = y^3, x = ty$, and the full blowup is given by
the now familiar parametric equations $y = t^2,x = t^3,t=t$ for all
$t\in K$. It is isomorphic to the affine line $K^1$; note that the
singularity at the origin has disappeared. (The alternative of imposing
the equation $y=ux$ rather than $x = ty$ leads to a more complicated
realization of the same variety $K^1$.) For a more interesting example,
consider the variety $W$ defined by the equation $y^2 = x^2(x+1)$, or
equivalently by $y^2 - x^2 - x^3 = 0$. Imposing the equation $y = tx$,
we get $t^2 x^2 - x^2 - x^3 = x^2(t^2 - 1 - x) = 0$, which factors,
having the two solutions $x=0$ or $t^2=x +1$. What happens here is that
the singular point $(0,0)$ in the original variety has a two-dimensional
tangent space; more precisely, we regard the lines $y = x$ and $y = -x$
as defining the two directions of that space, since $y^2 - x^2 =
(y-x)(y+x)$ is the homogeneous term of least degree in the equation
defining $W$. The singular point $(0,0)$ of $W$ in effect splits up into
two points $(0,-1)$ and $(0,1)$, interpreting the coordinates of these
points as values of $x$ and $t$ respectively. Both points are now
nonsingular on the strict transform and each has only one of the tangent
directions that were originally present at $(0,0)$. In another example
in homework, you will blow up the variety defined by $y^3 = x^5$; in
this case $(0,0)$ is still a singular point in the blowup, but one
further blowup resolves the singularity.

In general the preimage of the \lq\lq bad subset" $W$ of $V$ in the
blowup of $V$ corresponds to the algebra $G_I(A)$, so that it is a
projective variety with $G_I(A)$ as its homogeneous coordinate ring.
Suppose that $A$ is the coordinate ring of an affine variety $V$
containing 0, so that the corresponding ideal $J$ contains the
augmentation ideal $M=(x_1,\ldots,x_n)$ generated by the variables. For
each $f\in J$, define in$(f)$, the {\sl initial form} of $f$, by letting
$m$ be the largest index with $f\in M^m$ and then taking in$(f)$ to be
the image of $f$ in $M^m/N^{m+1}$; you should think of in$(f)$ as the
sum of the homogeneous terms of $f$ of least degree. The ideal
consisting of all in$(f)$ as $f$ runs through $J$ defines the so-called
{\sl tangent cone} of $V$; its coordinate ring is $G_M(A)$.

We conclude with some further remarks about minimal sets of generators
for prime ideals. We have seen that, given a prime ideal $P$ of height
$d$ in a Noetherian ring $A$, there are $x_1,\ldots,x_d\in P$ such that
$P$ is one of the finitely many minimal primes $P_1,\ldots,P_n$ over $J
= (x_1,\ldots,x_d)$. By choosing $x_{d+1}$ to lie in $P$ but none of the
other minimal primes $P_i$ we get a new ideal $J'= (x_1,\ldots,x_{d+1}$
such that the only minimal prime over $J'$ is $P$, whence the radical of
$J'$ must be $P$. Thus {\sl any prime ideal of height $d$ can be
  generated up to taking radicals by $d+1$ elements}, It is an open
question, even for ideals in polynomial rings, whether a prime ideal of
height $d$ is the radical of an ideal generated by $d$ elements. Perhaps
the most famous example is the ideal of the {\sl rational quartic curve}
in $\mathbf P^3$, which may be described as the kernel of the map
$K(t_0,t_1,t_2,t_3]\rightarrow K{s,t}$ sending $t_0$ to $s^4, t_1$ to
$s^3 t, t_2$ to $st^3$, and $t_3$ to $t^4$. Remarkably enough, it is
known in all positive characteristics that this ideal is the radical of
an ideal generated by two elements (which two depends on the
characteristic), but this is not known in characteristic zero! (In
geometric terms, the problem is not that the quartic curve is
intrinsically complicated; it is after all isomorphic to $\mathbf P^1$, but
it sits in $\mathbf P^3$ in a complicated way.)

\end{document}
