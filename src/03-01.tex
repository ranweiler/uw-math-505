\documentclass[10pt]{article}
\usepackage{amsmath, amssymb}

\begin{document}

\section*{Math 505, 3/1}

We now explore the behavior of dimension in more detail in the
projective setting, but before we do this we digress to generalize to
finitely generated modules $M$ over Noetherian rings $A$ our earlier
recipe attaching finitely many prime ideals of $A$ to any ideal $I$ of
it (the minimal primes over $I$). So let $M$ be such a module. The set
of annihilator ideals $A_m = \{x\in A: xm = 0\}$ as $m$ runs through teh
nonzero elements of $M$ must have a largest element, say $A_{m_1}$,
which we claim must be a prime ideal. Indeed, if $xym_1=0$ for some
$x,y\in A$ but $xm_1,ym_1\ne0$, then $A_{ym_1}$ is strictly larger than
$A_{m_1}$, a contradiction. Then the submodule $Am_1$ of $M$ generated
by $m_1$ takes the form $A/P_1$ for some prime ideal $P_1$. Moding out
by $Am_1$ and repeating this procedure, we find that the quotient
$M'=M/Am_1$ admits an element $m_2$ whose annihilator $P_2$ is another
prime ideal. Moding $M'$ out by $Am_2$ and continuing, we get a chain
$M_0=0\subset M_1\subset M_2,\subset\cdots$ of submodules of $M$ such
that eauch quotient $M_i/M_{i-1}\cong A/P_i$ for some prime ideal $P_i$
of $A$. But there are no strictly increasing chains of submodules of
$M$, so the above chain of submodules terminates at some $M_n = M$. Now
the ideals $P_1,\ldots,P_n$ arising from the chain are not uniquely
determined, but if $P$ is one of them and we localize $M$ at $P$
(letting $M_P$ consist of all formal fractions $m/s$ with $m\in M,s\in
A,s\notin P$, decreeing that $m/s = n/t$ if there is $u\notin P $ with
$u(tm - ns) = 0$ in $M$) then we find that all quotients $A/P_i$
disappear under this operation if $P_i$ is not contained in $P$, while
if $P_i = P$ then the localization is the fraction field of $A/P$, which
is also the quotient of the localized ring $A_P$ at its maximal ideal
$PA_P$. The upshot is that {\sl for any chain $M_0 =
  0\subset\cdots\subset M_n= M$ of submodules of $M$ as above with
  $M_i/M_{i-1}\cong A/P_i, P_i$ prime, then the ideals $P$ among
  $P_1,\ldots,P_n$ not containing any others are uniquely determined,
  each along with the number $\mu_P(M)$ of times it occurs as a $P_i$
  (and in fact $\mu_P(M)$ is the length of $M_P$ over the local ring
  $A_P$)}. We call the minimal ideals among the $P_i$ the {\sl
  associated primes} of $M$. Now if $A$ and $M$ both happen to be
graded, then we can carry out the above construction considering only
{\sl homogeneous} elements of $M$ throughout and observing under this
restriction all quotients of $M$ retain the grading. Thus we arrive at a
finite collection of graded prime ideals $P_1,\ldots,P_n$ attached to
$M$ whose minimal elements $P$ together with their multiplicities
$\mu_P(M)$ are uniquely determined by $M$.

Now if $V\subset\mathbf P^n$ is a projective variety and $M=S/I$ is its
homogeneous coordinate ring, then $M$ is in particular a graded
$S$-module, whence by the machinery developed last week it has a Hilbert
polynomial $p_M$ which is such that if $m\in\mathbf Z$ is sufficiently
large, then $p_M(m)$ equals the dimension over $K$ of the $m$th graded
piece $M_m$ of $M$. The degree $d$ of $p_M$ is then the dimension of $V$
(since we have seen that the same ideal $I$, viewed as an ordinary
radical ideal in $K[x_1,\ldots,x_{n+1}]$ has as its variety the affine
cone $C(V)$, which has dimension $d+1$). We also saw last week that the
leading coefficient of $p_M$ equals $r/d!$ with $r$ a positive integer;
we call it the {\sl degree} of $V$; unlike the dimension of $V$, this
number depends on the way $V$ is embedded in projective space and not
just on $V$ itself. If $V$ is reducible with irreducible components
$V_1,\ldots,V_r$, corresponding to the prime ideals $P_1,\ldots,P_r$
containing $I$, then the discussion in the above paragraph shows that
the degree of $V$ is given by $\sum_i \mu_{P_i}(S/I) d_i$, where $d_i$
is the degree of the variety corresponding to $P_i$, since here the
$P_i$ are exactly the minimal primes containing $I$. Now the Hilbert
polynomial $p_S(m)$ of $S$ itself is easily computed to be the binomial
coefficient $m+n\choose n$, so $\mathbf P^n$ itself has degree 1. A
hypersurface $H$ in $\mathbf P^n$ defined by a single homogeneous
polynomial of degree $d$ has Hilbert polynomial $p_S(m) - p_S(m-d)$
whence it has degree $d$ as well. The union of two varieties $X_1,X_2$
of the same dimension $d$ such that $X_1\cap X_2$ has dimension less
than $d$ has degree the sum of the degrees of $X_1$ and $X_2$. Finally,
suppose we take the intersection $Y\cap H$ of an irreducible variety
$Y\subset\mathbf P^n$ of dimension $d$ and a hypersurface $H$ not
containing it. We have seen that the irreducible components
$Z_1,\ldots,Z_r$ of $Y\cap H$ all have dimension $d-1$; defining the
{\sl intersection multiplicity} $i(Y,H;Z_j)$ to be the length
$\mu_{P_j}(S/(I_Y + I_H)$, where $I_y.I_h$ are the ideals corresponding
to $Y,H$ and each $P_j$ is the prime ideal corresponding to $Z_j$, then
 $$
 \sum_{j=1}^r i(Y,H;Z_j) = d_Y d_H
 $$
 \noindent where $d_Y,d_H$ are the respective degrees of $Y$ and $H$. In
 particular, {\sl two curves in $\mathbf P^2$, defined by homogeneous
   polynomials of degrees $d,e$, possibly reducible but having no
   irreducible component in common, intersect in exactly $de$ points if
   these are counted with appropriate multiplicities}. Thus curves in
 $\mathbf P^2$, in stark contrast to curves in $K^2$, intersect in a
 very nice and uniform way.

\end{document}
