\documentclass[10pt]{article}
\usepackage{amsmath, amssymb}

\begin{document}

\section*{Math 505, 2/1}

Now (and for the rest of the course) we broaden our focus to commutative
rings $R$, usually however assumed Noetherian. A special case of
particular importance for us is the one where $R=K[x_1,\ldots,x_n]$, the
polynomial ring in $n$ variables over a field $K$ (usually taken to be
algebraically closed for simplicity), or a quotient of this ring. We
have seen that every nonzero ideal in a Dedekind domain is uniquely a
product of prime ideals; for general commutative rings this is too much
to expect, but we can still hope to get a better grasp on prime ideals
than on arbitrary ones. We will therefore focus on prime ideals in what
follows.

Start with the particular example $R=K[x_1,\ldots,x_n]$ mentioned above,
where $K$ is an algebraically closed field. If $n=1$, we know that the
nonzero prime ideals in $R$ are all generated by single linear
polynomials $x-a$ for some $a\in K$, and that every such ideal is
maximal. It is natural to wonder what happens for larger $n$. To this
end, we define {\sl affine algebraic variety $V$} in $K^n$ to be the
subset $S$ of common zeros of some nonempty collection $\mathcal S$ of
elements of $R$. Since the common zeros of the polynomials in $\mathcal
S$ are the same as those of the ideal $I$ generated by it, we may assume
that $\mathcal S$ is in fact an ideal $I$ of $R$; denote the variety of
its common zeros by $V(I)$ and call the quotient ring $R/I$ the {\sl
  coordinate ring} of $V(I)$; we denote this ring by $K[V]$. We will see
later that $K[V]$ depends only on $V$ (as the notation indicates) if the
ideal i$I$ s suitably restricted; for now we have a map $I\rightarrow
V(I)$ from ideals of $R$ to subsets of $K^n$, but this map is clearly
not a bijection; even for $n=1$, the varieties $V(x),V(x^2)$ of the
respective principal ideals generated by $x,x^2$ are both the point
$\{0\}$. For $n>1$, it is not even obvious that $V(I)$ is nonempty if
$I$ is proper.

To better understand $V(I)$ we focus on $K[V]$; this is generated as
$K$-algebra (that is, as a ring containing a copy of $K$ which in turn
contains its identity element) by finitely many elements
$x_1,\ldots,x_n$. I now claim that {}\sl given any finitely generated
algebra $A$ over $K$, there are finitely many elements
$y_1,\ldots,y_m\in A$ that are algebraically independent over $K$ such
that $A$ is a finitely generated integral extension of
$B=K[y_1,\ldots,y_m]$, that is, that every element of $A$ satisfies a
monic polynomial equation with coefficients in $B$. We prove this by
induction on $n$. If the $x_i$ are already algebraically independent
then the result is clear; otherwise we have a polynomial $p$ in the
$x_i$ with coefficients in $K$ that equals 0 in $A$. We may regard $p$
as a polynomial in just the last variable $x_n$ (renumbering if
necessary) with coefficients polynomials in the other variables $x_i$.
Let $d$ be the maximum degree of all of these coefficients. We now make
a change of variable, setting $x_i = y_i + x_n^{(d+1)^i}$ for $i<n$.
Writing out $p$ as a polynomial in $y_1,\ldots,y_{n-1},x_n$ we find that
every monomial term of every coefficient of $p$ gives rise to a
different power of $x_n$; the top power of $x_n$ occurring has constant
coefficient $c$ and arises from the lexicographically highest term
$cx_1^{m_1}\ldots x_{n_1}^{m_{n-1}}$ of any coefficient of $p$, that is,
one first of all with the highest possible power of $x_{n-1}$, then
among these one with the highest possible power of $x_{n-2}$, and so on;
if two coefficients appear with identical powers of the $x_i$ for $i<n$,
then the one we want is the one attached to the higher power of $x_n$.
Dividing by $c$, we get a monic polynomial in $x_n$ with coefficients
polynomials in the $y_i$, so that $A$ is integral over the subalgebra
generated by $K$ and the $y_i$. By induction we realize $A$ in the
desired form. Now we pause to note a simple ring-theoretic fact: {\sl
  given two integral domains $A\subset B$ with $B$ integral over $A$,
  then $B$ is a field if and only if $A$ is}. Indeed, if $A$ is a field
and $x\in B,x\ne0$, then we have an equation $x^n = \sum_{i=0}^{n-1} a_i
x^i$ with $a_i\in A$; cancelling out a suitable power of $x$, we may
assume that $a_0\ne0$, and then $a_0$ is a multiple of $x$ and so has a
multiplicative inverse, whence $x$ does too. Conversely, if $B$ is a
field and $x\in A,x\ne0$, then we have an equation $x^{-n} =
\sum_{i=0}^{n-1} a_i x^{i-n}$; multiplying by $x^{n-1}$ we realize
$x^{-1}$ as a polynomial in $x$ with coefficients in $A$, whence it lies
in $A$ as desired. Now given a proper quotient $R/I$ of $R/I$ that is a
field $K'$, we deduce that $K'$ must be an integral extension of $K$
itself (as opposed to $K[y_1,\ldots,y_m]$, which is never a field for
$m>0$), whence if $K$ is algebraically closed , we must have $K'=K$.
This forces the generators $x_i$ of $R$ to map to elements $a_i$ of $K$
in the canonical map from $R/I$ to $K'=K$, whence {\sl every maximal
  ideal of $R$ takes the form $(x_1-a_1,\ldots,x_n-a_n)$, if $K$ is
  algebraically closed}. Since every proper ideal $I$ of $R$ is in turn
contained in a maximal one, we deduce as desired that \sl{$V(I)$ is
  nonempty for every proper ideal $I$ of $R=K[x_1,\ldots,x_n]$}. This is
the weak form of a famous result called the Nullstellensatz (zero places
theorem); we will prove the strong form and deduce a bijection between
suitably restricted ideals $I$ and varieties next time.

\end{document}
