\documentclass[10pt]{article}
\pdfinfoomitdate=1
\pdftrailerid{}
\usepackage{amsmath, amssymb}

\begin{document}

\section*{Math 505, 1/20}

Now we digress a bit to give the formulas for solving cubic and quartic
equations. As mentioned before, we will step back in time about five
centuries, solving these equations using only high-school algebra
techniques and avoiding any field or group theory. Given the cubic
equation $x^3 + ax^2 + bx + c = 0$, we begin by changing the variable,
setting $y = x - (a/3)$. Equating $(y- (a/3))^3 = a(y - (a/3)^2 + b(y -
(a-3)) + c$ to 0 and collecting coefficients of the powers of $y$, we
find that the coefficient of $y^2$ cancels. Rewrite the resulting
equation as $y^3 + py + q = 0$; we call the left side a {\sl reduced
  cubic}. Now make the substitution $y = z + (k/z)$, where $k$ will be
specified in a moment. We get $z^3 + (3k + p)z + (3k^2 + q + pk)z^{-1} +
k^{-3}z^{-3} = 0$. Two terms drop out by making the choice $k = -p/3$;
with this choice we get $z^3 + q - (p^3/27)z^{-3} = 0$. This last
equation becomes quadratic in $z^3$ if multiplied by $z^3$ on both
sides. Solving it by the quadratic formula and plugging back into $y$,
we get $y = ((-q + S)/2)^{1/3} + ((-q - S)/2)^{1/3}$, where $S =
\sqrt{q^2 + (4p^3/27)}$ and the cube roots must be chosen so that their
product is $-p/3$ (but otherwise they are unrestricted). This gives us
exactly three roots, as desired; note that if $p,q$ are real and $q^2 +
(4p^3/27)$ is negative, then the roots are real, but the expressions
that lead to them involve complex numbers. This turns out to be
unavoidable; it was in fact one of the motivations that led
mathematicians to accept complex numbers, since they can be required to
express even real solutions to certain equations. (If conversely $q^2 +
4p^3/27$ is positive, then you can check that one root is real while the
other two are complex.) If $p,q$ are rational, then the quantity
$D=-27q^2 - 4p^3$, a rational multiple of $q^2 + 4p^3/27$, turns out to
determine the Galois group of $x^3 + px + q$ over $\mathbf Q$; assuming
this polynomial has no rational root, or equivalently is irreducible
over $\mathbf Q$, then its Galois group is $A_3$ or $S_3$ according as
$D$ is the square of a rational number or not. We call $D$ the {\sl
  discriminant} of the polynomial; we will see below that any polynomial
of degree $n$ has a discriminant which behaves similarly to $D$ in that
it determines whether the Galois group is a subgroup of $A_n$ or not.

Turning now to the quartic equation $x^4 + ax^3 + bx^2 + cx + d = 0$,
our first step is to get rid of the $x^3$ term, this time by making the
substitution $y = x - a/4$. Combining like powers of $y$ we now get $y^4
+ py^2 + qy + r = 0$ fro some $p,q,r$. Once again we introduce a new
parameter, this time called $t$ and add and subtract $t^2/4$, rewriting
the resulting equation as $(y^2 + (t/2))^2 - [y^2(t-p) - qy + ((t^2/4) -
  r)] = 0$. We now choose the parameter $t$ in such a way that the
quadratic polynomial in brackets becomes a (constant times a) perfect
square, by making its discriminant $q^2 - 4(t-p)((t^2/4) - r)$ equal to
0. This is a cubic equation, called the {\sl resolvent cubic}, which we
can solve by the above paragraph. Now we can rewrite our original
equation as $A^2 - B^2 = 0$, for suitable expressions $A,B$; equating
$A$ to $\pm B$ and solving, we see that we can find the four roots of
our original equation by solving two quadratics. We will not push this
through to get explicit expressions for the roots, but note that neither
the resulting formula not the cubic formula derived above could have
been derived just by analyzing the groups $S_3$ and $S_4$; we knew in
advance that some radical formula had to exist, but Galois theory alone
does not tell us what it is, though it does correctly predict that
expressing the roots of a quartic polynomial requires solving a cubic
polynomial along the way, as $\mathbf Z_3$ is one of the cyclic
subquotients of $S_4$.

Given a polynomial $p$ of degree $n$ over any field $K$ with roots
$r_1,\ldots,r_n$ in its splitting field, we note that the product
$D=\Pi_{i<j} (r_i - r_j)$ of all the root differences remains unchanged
if an even permutation is applied to the $r_i$, while it changes by a
sign if an odd permutation is applied to the $r_i$. It follows that
$D^2$ always lies in the basefield $K$ (being fixed by the Galois group,
a subgroup of $S_n$) snd that $D$ lies in $K$ if and only if the Galois
group of $p$ is a subgroup of $A_n$ (i.e. consists entirely of even
permutations of the $r-I$). By expressing $D^2$ in terms of the
coefficients of $p$ (which in principle can always be done), we thereby
derive a criterion for the Galois group of $p$ to be a subgroup of
$A_n$. If $p$ is monic and quadratic, so equal to $x^2 + bx + c$, then
$D^2$ is the familiar expression $b^2 - 4c$ from high-school algebra,
which among other things determines whether the roots of $p$ are real or
not (if $K$ = $\mathbf R$). If $p$ is a monic reduced cubic, then $D^2$
turns out to equal $-27q^2 - 4p^3$, as mentioned above. There is a
similar but more complicated formula for the discriminant of a reduced
quartic polynomial.

We conclude by mentioning a purely numerical criterion for solvability
of a quintic polynomial $p$ over $\mathbf Q$. There is a polynomial in
the roots $r_1,\ldots,r_5$ of a such a polynomial (regarded as
independent variables) such that the subgroup of $S_n$ fixing this
polynomial is the normalizer $N$ of the cyclic subgroup generated by a
5-cycle of the roots $r_i$ (where as above $S_5$ acts on the $r_i$ by
permutations). This polynomial has exactly 6 distinct images
$i_1,\ldots,i_6$ under $S_5$(since $N$ has index 6 in $S_5$. Taking the
product $q$ of the $x-i_j$ we get a polynomial of degree 6 over $\mathbf
Q$ whose coefficients are polynomials in the coefficients of $p$. This
polynomial has a rational root if and only if the Galois group of $p$
lies in a conjugate of $N$ in $S_5$; this turns out to be the case if
and only if this Galois group is solvable. Thus $q$ has a rational root
if and only if $p$ is solvable by radicals over $\mathbf Q$. A formula
for $q$ is given on p. 639 of the third edition of Dummit and Foote; it
takes almost half a page to write out.

\end{document}
