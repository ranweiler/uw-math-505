\documentclass[10pt]{article}
\usepackage{amsmath, amssymb}

\begin{document}

\section*{Math 505, 1/25}

We now take up the study of commutative rings, starting with a
particular class of them rather than such rings in general. Rings in
this class are a little but not much more complicated than PIDs in some
sense, and in particular finitely generated modules over them admit a
very similar classification to the PID case. These rings are called {\sl
  Dedekind domains}; we will begin with a quite nonstandard definition
of them which is however equivalent to the standard one and most
convenient for our purposes. Call a (commutative) integral domain $R$ a
Dedekind domain if for any ideals $I,J$ of $R$ with $I\subset J$ there
is an ideal $K$ of $R$ with $I = JK$. We note at once that the ideal $K$
is unique, except for the trivial case $I = J = 0$; indeed, if $JK =
JK'\ne0$, then we can choose $a\in J$ with $a\ne0$ and then there is an
ideal $L$ with $LJK = LJK' = aK = aK'$, forcing $K = K'$ since $R$ is an
integral domain. We say that nonzero ideals $I,J$ of $R$ lie in the same
{\sl ideal class} if there are nonzero $a,b\in R$ with $aI = bJ$; one
checks immediately that this is an equivalence relation and the ideals
equivalent to $R$ itself are exactly the principal ones. Then the ideals
$J$ in the same class as $I$ are exactly those isomorphic as $R$-modules
to $I$, for if $\pi:I\rightarrow J$ is an isomorphism and $x\in
I,x\ne0$, then we have $x\pi(y) = \pi(xy) = \pi(x)y$ for $y\in I$,
whence $xJ = \pi(x)I$. Note that ideal classes form a group under
multiplication; the definition of Dedekind domain guarantees that every
ideal class has an inverse. Next we show that every ideal $I$ in a
Dedekind domain is finitely generated. If $I\ne0$, choose a nonzero
element $a$ of it and write $(a) = IJ$ for some ideal $J$. By definition
of the ideal product, there are finitely many elements $x_1,\ldots,x_n$
in $I, y_1,\ldots,y_n$ in $J$ with $a = x_1 y_1 +\ldots+ x_n y_n$. The
ideal $I'$ generated by the $x_i$ then satisfies $I'J\subset IJ = (a)$,
but it contains the generator $a$, so $IJ = I'J$, whence $I = I'$ is
finitely generated. From this it follows that $R$ is {\sl Noetherian}:
every nonempty set of ideals of $R$ has a maximal element, or
equivalently there are no infinite strictly ascending chains $I_1\subset
I_2\subset\cdots$ of ideal in $R$: indeed, given such a chain, its union
$I$ would be an ideal and this finitely generated; but then for $i$
large enough $I_i$ contains all generators of $I$ and thus $I$ itself,
whence $I_i = I_j$ for $j>i$, a contradiction. Next, I claim that {\sl
  every nonzero ideal $I$ of $R$ is uniquely (up to reordering) a finite
  product of maximal ideals}. Given $I$, if it is not maximal already,
then it lies in a maximal ideal $M_1$, whence $I= M_1 I_2$ for an ideal
$I_2$ strictly containing $I$; if $I_2$ is not maximal, we write $I_2 =
M_2 I_3$ for some maximal ideal $M_2$ and ideal $I_3$ strictly
containing $I_2$, and so on; since there are no infinite strictly
ascending chains of ideals in $R$, the process terminates, realizing $I$
as a finite product of maximal ideals $M_1\cdots M_n$. For the
uniqueness, observe that if also $I = N_1\cdots N_r$ with the $N_i$
maximal, then $N_1$, as a prime ideal of $R$, contains the product
$M_1\cdots M_n$ and thus $M_i$ for some $i$, whence by maximality $N_1 =
M_i$. Cancelling $M_i, N_1$ from both products equalling $I$, we see
that $n=r$ and the $N_i$ are a permutation of the $M_j$ (note that the
$M_i$ need not be distinct, and similarly for the $N_j$). As a
consequence, {\sl every nonzero prime ideal $P$ of $R$ is maximal}, for
$P$ must be a product $M_1\cdots M_n$ of maximal ideals, whence $P$
contains and must equal $M_i$ for some $i$. Finally, one last
ideal-theoretic property of $R$: {\sl given any nonzero ideal $I$ of $R$
  and $f$ in the quotient field of $R$, if $fI\subset I$, then $f\in
  R$.} Indeed, if $fI\subset I$, then $fI$ is an ideal of $R$ contained
in $R$, whence $fI = IRf = IJ$ for some ideal $J$ of $R$, whence by
cancellation the $R$-submodule $Rf$ of $F$ generated by $f$ equals $J$,
forcing $f$ to lie in $R$. We express this trio of properties ($R$ a
Noetherian integral domain, every nonzero prime ideal of $R$ is maximal,
any $f$ in the quotient field of $R$ that maps a nonzero ideal to itself
actually lies in $R$) by saying that $R$ is an {\sl integrally closed
  Noetherian domain of dimension one}; here \lq\lq integrally closed"
means that every element of the quotient field $F$ of $R$ satisfying a
monic polynomial with coefficients in $R$ actually lies in $R$, a
property equivalent to the third one above; while \lq\lq dimension one"
expresses the maximality of every nonzero prime ideal; we will give a
general account of dimension for commutative rings later. The standard
definition of Dedekind domain is then an integrally closed Noetherian
domain of dimension one.

Now we are ready to exhibit our main example of Dedekind domains. Let
$L$ be an algebraic number field (a finite extension of $\mathbf Q$, not
necessarily Galois). Let $R$ consist of the algebraic integers in $L$
(roots of monic polynomials over $\mathbf Z$). Then we claim that $R$ is
a Dedekind domain. This will take some work to prove. Begin by noting
(as we did when discussing characters of finite groups) that $R$ is at
least a ring. Next we claim that $R$ is a free $\mathbf Z$-module of
finite rank. To prove this note first that by clearing denominators we
see that for any $x\in L$ there is a nonzero $n\in\mathbf Z$ with $nx\in
R$. Let $x_1,\ldots,x_m$ be a basis of $L$ over $\mathbf Q$ and for each
$i$ choose a nonzero integer $n_i$ with $y_i = n_i x_i\in R$, so that
the $y_i$ also form a basis of $L$ over $\mathbf Q$. For any $y\in R$,
the products $yy_i$ lie in $R$ and therefore have traces that are
rational algebraic integers and thus integers. But any $y\in L$ is
completely determined by the traces of the $yy_i$, since the trace is a
nondegenerate (symmetric) bilinear form on $L$. It follows that as an
additive group $R$ is isomorphic to a subgroup of $\mathbf Z^m$, whence
it like $\mathbf Z^m$ is free and finitely generated over $\mathbf Z$
(in fact of rank exactly $m$).

We now claim that {\sl every nonzero proper ideal $I$ in $R$ contains a
  product of nonzero prime ideals}. If this failed, there would be a
largest ideal $I$ for which it failed, so that $I$ contains no such
product, but any larger ideal does contain such a product. But then $I$
cannot itself be prime; if $a,b\in R$ are such that $ab\in I$ but
$a,b\notin I$, then the larger ideals $I+(a),I+(b)$ of $R$ contain
products of nonzero prime ideals, whence so too does their product
$(I+(a))(I+(b))$, which lies in $I$, a contradiction. Now we need our
last property above of any nonzero ideal $I$ of $R$ : if an element $f$
of the quotient field of $R$ satisfies $fI\subset I$, then $f\in R$. To
see this we note that $I$ is necessarily a free module of finite rank
over $\mathbf Z$; under the hypothesis multiplication by $f$ defines a
$\mathbf Z$-linear transformation from this module to itself, whence by
the Cayley-Hamilton Theorem it satisfies its own characteristic
polynomial, which is monic with integer coefficients, forcing $f$ to lie
in $R$, as desired. We will complete the proof that $R$ is a Dedekind
domain next time.

\end{document}
