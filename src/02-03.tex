\documentclass[10pt]{article}
\usepackage{amsmath, amssymb}

\begin{document}

\section*{Math 505, 2/3}

We now know that every proper ideal $I$ of $R=K[x_1,\ldots,x_n]$ has a
nonempty variety $V(I)$ of common zeros, but the map from $I$ to $V(I)$
is not 1-1, even if $n=1$. The best we can do in that case is to send
the ideal $I$ generated by the product of {\sl distinct} linear
polynomials $x-a_1,\ldots,x-a_m$ to the finite subset
$\{a_1,\ldots,a_m\}$ of $K$; this is a bijection between certain ideals
of $R$ and all (affine algebraic) varieties. To define the
higher-dimensional analogue of begin generated by a product of distinct
linear polynomials, we need a general ring-theoretic notion Given any
ideal $I$ in a commutative ring $A$, its {\sl radical $\sqrt{I}$}
consists by definition of all $x\in A$ such that $x^n\in I$ for some
$n$; commutativity of $A$ and the binomial theorem guarantee that
$\sqrt{I}$ is indeed an ideal and $\sqrt{\sqrt{I}} = \sqrt{I}$. We call
the ideal $I$ {\sl radical} if it equals its own radical. Now we can
finally specify precisely which ideals $I$ of $R$ we will focus on,
namely the radical ones. Then the strong form of the Nullstellensatz
proved last time states that {\sl if $f\in R$ is such that $f(v) = 0$
  for all $v\in V(I)$, then $f\in\sqrt{I}$; thus the map $I\rightarrow
  V(I)$ implements an inclusion-reversing bijection between proper
  radical ideals of $R$ and varieties in $K^n$}. To prove this, we need
a fact about radicals of ideals in general commutative rings: {\sl the
  radical of an ideal $I$ in a ring $A$ is the intersection of all prime
  ideals containing $I$}. Indeed, any prime ideal $P$ containing $I$
contains any element of its radical, by definition of prime ideal;
conversely, if no power of $x$ lies in $I$, let $J$ be an ideal of $A$
maximal with respect to exclusion of all powers of $x$. Then $J$ is
prime, for any ideals $J_1,J_2$ properly larger than $J$ contain powers
of $x$, say $x^{n_1},x^{n_2}$, but then $J_1 J_2$ contains $x^{n_1+n_2}$
and so does not lie in $J$. Returning now to the polynomial ring $R$,
suppose that no power of $f\in R$ lies in the ideal $I$ and let $P$ be a
prime ideal of $R$ containing $I$ but not $f$. Then $P$ generates a
proper prime ideal $PR_f$ in $R_f$, the localization of $R$ at all
powers of $f$; enlarging this to a maximal ideal $M$ of $R_f$, we know
that the quotient of $R_f$ by $M$ must be isomorphic to $K$, whence we
get a point $(a_1,\ldots,a_n)$ in $K^n$ at which all polynomials in $I$
vanish but $f$ does not, as desired. (Note also that the variety $V(I)$
of any ideal coincides with the variety $V(\sqrt{I})$ of its radical, so
we don't lose any varieties by restricting to radical ideals.)

Now that we finally know that our map $I\rightarrow V(I)$ is a bijection
if the ideal $I$ is suitably restricted, we can record some simple
properties of this map. Clearly the variety $V$ of a sum $\sum I_i$
ideals (even an infinite sum) is the intersection of the varieties
$V(I_i)$ of the $I_i$, while the variety $V(I_1\cdots I_m)$ is the union
of the $V(I_i)$. Likewise the variety $V(I_1\cap\cdots\cap I_m)$ of the
intersection of the $I_i$ is the union of the $V(I_i)$; in fact the
intersection of the $I_i$ is exactly the radical of their product, if
the $I_i$ are radical. This says exactly that the varieties $V(I)$ of
all ideals in $R$ (including $R$ itself) obey the axioms for the {\sl
  closed} sets in a topology on $K^n$, which we call the {\sl Zariski
  topology}; the open sets in this topology are of course the
complements of the closed ones. If $K=\Bbb C$, then every Zariski-open
subset is also open in the Euclidean topology; but the converse is quite
false; in a nutshell, the difference is that nonempty Zariski-open
subsets of $\Bbb C^n$ have to be very \lq\lq fat", in the sense that
they are dense in both the Zariski and Euclidean topologies. For $n=1$,
the Zariski-closed subsets of $K$ are precisely the finite subsets
together with $K$ itself; in general, points in $K^n$ are always closed
in the Zariski topology, but this topology is not Hausdorff. It has a
very special property with no counterpart in the Euclidean topology,
namely that $K^n$ is {\sl Noetherian}: there are no strictly descending
infinite chains of closed subsets (since there are no strictly ascending
chains of ideals in $R$, as you will prove in homework next week). It
follows that every closed set in this topology is a finite union of {\sl
  irreducible} closed sets, none of these being the union of two closed
proper subsets. (If there were any closed subsets that are not finite
unions of irreducible sets, there would be a smallest such closed set
$S$, which would have to be reducible; but then both of the proper
subsets of $S$ would be finite unions of irreducible sets and so $S$
would be also, a contradiction.) The irreducible subvarieties whose
union is a given variety are called its {\sl irreducible components};
they are unique if we impose the natural requirement that none of them
be contained in another. They behave somewhat like connected components
of a topological space, but are {\sl not} disjoint, in general; for
example, the variety defined by the equation $xy = 0$ in $K^2$ is the
union of the coordinate axes, which intersect at the origin. More
generally, the variety $V(p)$ of a nonzero principal ideal $(p)$ in $R$
has as its irreducible components $V(p_i)$, where the $p_i$ are the
unique monic irreducible factors of $p$ in $R$ (recall that $R$ is a
unique factorization domain). Since we have seen that the variety of the
intersection of radical ideals is the union of the varieties of the
ideals, we deduce that {\sl every radical ideal in $R$ is a finite
  intersection of prime ideals, where these are unique if none of them
  is allowed to contain another}. We also see that {\sl a variety $V(I)$
  is irreducible if and only if its coordinate ring is an integral
  domain, or equivalently if and only if $\sqrt{I}$ is prime.} In the
literature varieties are often taken to be irreducible by definition.

\end{document}
