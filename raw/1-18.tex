We now prove the other half of the Galois Criterion:  if a polynomial $p$ over a field $K$ of characteristic 0 has solvable Galois group, then $p$ is solvable by radicals.  The proof has a number of parallels with the proof of its converse, but also involves an important result from last quarter that was proved in two different ways; it is this result that requires the assumption of characteristic 0.  So suppose that the Galois group $G$ of the polynomial $p$ over the field $K$ is solvable.  Applying the Galois correspondence we see that there is a chain of fields
$K=K_0\subsetK_1\subset\cdots\subset K_n$ with each $K_i$ cyclic (Galois with cyclic Galois group) over $K_{i-1}$ and $K_n$ containing the splitting field of $p$ over $K$.  Let $d_i$ be the degree of
$K_i$ over $K_{i-1}$ for $1\le i\le n$ and let $N$ be the product of the $d_i$.  We now tweak our chain of fields, as we did in proving the converse result, by adding roots of 1:  let $K_0'$ be the splitting field of $x^N - 1$ over $K_0$ and inductively $K_i'$ the splitting field of $p_i$ over $K_{i-1}'$ for $i>0$, where $K_i$ is the splitting field of $p_i$ over $K_{i-1}$.  Then any $K_{i-1}'$-automorphism of $K_i'$ restricts to a $K_{i-1}$-automorphism of $K_i$ and is determined by this restriction, since the roots of $p_i$ generate $K_i'$ over $K_{i-1}'$.  Hence the Galois group of $K_i'$ over $K_{i-1}'$ is a subgroup of the Galois group $\Bbb Z_{n_i}$ of $K_i$ over $K_{i-1}$ and so in particular is cyclic of order $m_i$ dividing $n_i$.  Now we invoke the result from last quarter:  given any field $F$ containing a primitive $m$th root of 1 and a cyclic extension $E$ of it of degree $m$, we must have $E=F(\alpha)$ for some $\alpha$ with $\alpha^m\in F$.  (We proved this in two different ways, first by using the rational canonical form of a matrix and later vis the cohomology of cyclic Galois groups.  Note that it fails for extensions of degree $p$ in characteristic $p$, which can be generated by a root of the polynomial $x^p - x - \beta$ for some $\beta$, rather than of $x^p - \beta$.)  The hypothesis is satisfied by each of our fields $K_i'$, since $K_i$ contains a primitive $N$th root and thus also a primitive $m_i$th root of 1, so each $K_i'$ is a radical extension of $K_{i-1}'$ for $i\ge1$; likewise $K_0'$ is clearly a radical extension of $K_0$, being generated by a primitive $N$th root of 1.  Since $K_n'$ contains $K_n$, which in turn contains the splitting field of $p$, we are done:  $p$ is solvable by radicals.

We don't have to confine ourselves to the basefield $\Bbb Q$.  Let $K$ be any field of characteristic 0 and let $L$ be the field of rational functions in $n$ variables $x_1,\ldots,x_n$ over $K$.  The symmetric group $S_n$ acts on the variables $x_i$ by permutations and thereby on $L$ by automorphisms, whence $L$ is Galois over its fixed field $L^{S_n}$, with Galois group $S_n$.  For $n=3$ or 4, then, the polynomial $(x-x_1)\cdots(x-x_n)$ must be solvable by radicals over $L^{S_n}$.  Radical formulas for its roots (in the two cases $n=3$ or 4) amount to two {\sl universal} formulas (analogous to the quadratic formula), one for the roots of any cubic polynomial over $K$, the other for any quartic polynomial.  We will derive these formulas (in a naive but historically accurate) way later, without using any field theory.

How does one go about computing Galois groups of polynomials in general and thereby deciding whether they are solvable by radicals? (This question was a major concern to the referees of Galois's original paper, which Galois could not fully address; his paper was rejected.)  We look at some simple examples, which turn out to be richer and less predictable than one might expect.  Take first the polynomial $x^3 - 2$ over $\Bbb Q$.  The splitting field is generated by two elements over $\Bbb Q$, namely the real cube root $2^{1/3}$ of 2 and a primitive cube root $\omega$ of 1, so its degree over
$\Bbb Q$ is the maximum possible one of 6, and the Galois group is the largest possible one for any cubic, namely $S_3$.  Now look at $x^4 - 2$ over $\Bbb Q$.  The splitting field is generated by $2^{1/4}$, the positive real fourth root of 2, and $i$ (a primitive fourth root of 1), so has degree 8 over $\Bbb Q$.  Any automorphism of this field permutes the four root of this polynomial; it can do so cyclically (by fixing $i$), or by a reflection of the square formed by these roots in the complex plane (via complex conjugation).  Hence the Galois group is the dihedral one $D_4$ of order 8.  It looks like we have a pattern here; in both cases the Galois group of the polynomial happens to coincide with the symmetry group of the regular polygon formed by the roots in the complex plane.  Alas, this pattern is broken already for $x^5 - 2$; it is not difficult to check that the splitting field has degree 20 over
$\Bbb Q$ in this case, so the Galois group is too big to be the dihedral group of order 10.  Matters are even worse for the polynomials $x^8 - 2$ and $x^8 - 3$:  the first of these has Galois group of order 16 but {\sl not} isomorphic to the dihedral group of this order, while the second has Galois group of order 32.  (The discrepancy arises because 2 is "closely related" in some sense to a primitive 8th root of 1 while 3 is not.)  