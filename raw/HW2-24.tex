\input amstex
\input amssym.def
\loadmsbm
\magnification=\magstep1
\centerline{\bf HW \#7, DUE 2-24}
\bigskip
\noindent 1. Show that the homogeneous coordinate ring of the projective variety $V$ defined by the equation $xz = y^2$ is not isomorphic to the homogeneous coordinate ring $K[x,y]$ of $\Bbb P^1$, even though $V$ and $\Bbb P^1$ are isomorphic.
\vskip .5in
\noindent 2. Decide whether the map from $\Bbb P^2$ to itself sending $(x,y,z)$ to $(xy,xz,yz)$ is a morphism or just a rational map.  Exhibit a rational inverse of this map.
\vskip .5in
\noindent 3. Show that $K^1$ and $\Bbb P^1$ (both given the Zariski topology) are homeomorphic, but not $K^n$ and $\Bbb P^n$ for $n>1$.  (Note that the dimension of an irreducible subvariety $V$ of $K^n$ or $\Bbb P^n$ can be defined purely topologically, as the length of the longest strictly increasing chain of subvarieties ending at $V$.  Find an intrinsic property of pairs of subvarieties of given dimensions that $\Bbb P^n$ has but $K^n$ lacks.)
\vskip .5in
\noindent 4. Let $A$ be any commutative ring.  Show that $\dim A + 1\le\dim A[x]\le 2\dim A + 1$, where $A[x]$ as usual is the polynomial ring in one variable over $A$.
\vskip .5in
\noindent 5. Show that the strong Nullstellensatz fails for the ring $C$ of continuous real-valued functions on the unit interval $[0,1]$, by exhibiting a radical nonmaximal ideal $I$ of this ring whose variety $V(I)$ is a single point.
\end