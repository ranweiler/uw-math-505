We continue with the setup from last time:  $L$ is a finite Galois extension of a field $K$ with Galois group $G$.  We head towards the {\sl Galois correspondence}, which establishes a bijection between subfields of $L$ containing $K$ and subgroups of $G$.  Let $L'$ be such a subfield.  Then $L$, as the splitting field of some polynomial $p$ over $K$ with no multiple roots, continues to be the splitting field of the same $p$ over $L'$, so that $L$ is Galois over $L'$.  We have seen that any
$\alpha\in L'$ with $\alpha\notin K$ is a root of some nonlinear polynomial $q$ over $K'$, all of whose roots lie in $L$ and are conjugate to $\alpha$ under a $K$-automorphism of $L$.  It follows that we can recover $K'$ as the set of all $x\in L$ fixed by Aut$_(K',L)$, and this group is a subgroup $H$ of
$G=$Aut$_K(L)$.  Hence {\sl the subfields $L'$ of $L$ containing $K$ all take the form
$L^H = \{x\in L: hx = x, h\in H\}$; in particular, there are only finitely many of them}.  The same result holds even if $L$ is only a finite separable extension of $K$, for then $L$ lies in a finite Galois extension $M$ of $K$ (the splitting field of a suitable product of polynomials over $K$, one for each element of a $K$-basis of $L$).  Now we want to see that there are {\sl exactly} as many subfields $L'$ as subgroups of $G$ if $L$ is Galois; we will prove this in steps, deriving other results interesting in their own right along the way.  We first show that $L$ is a {\sl simple} extension of $K$, generated by a single element $\alpha$.  To see this, note first that it is immediate if $K$ and $L$ are finite, for then the multiplicative group $L^*$ is cyclic (as you will prove in homework this week) and a generator also generates $L$ as an extension of $K$.  If $K$ is infinite and $\alpha_1,\alpha_2\in L$, then the subfield $L'=K(\alpha_1,\alpha_2)$ of $L$ generated by $K$ and $\alpha_1,\alpha_2$ has only finitely many fields between it and $K$, so there are distinct
$c,d\in K$ with $K(\alpha_1+c\alpha_2) = K(\alpha_1+d{\alpha_2)$; this forces 
$(d-c)\alpha_2,\alpha_2$, and $\alpha_1$ to lie in $K(\alpha_1+c\alpha_2)$, so that the single element $\alpha_1+c\alpha_2$ generates $L'=K(\alpha_1,\alpha_2)$.  Iterating this result with a finite basis $\ell_1,\ldots,\ell_n$ of $L$ as a $K$-vector space, we see that a suitable linear combination of the $\ell_i$ generates $L$ as an extension of $K$, as claimed.  Now let $E$ be any field and $H$ any finite group of automorphisms (not assumed to be $F'$-automorphisms for any particular subfield $F$ of $E$ for the moment).  If $\alpha\in E$ and if $\alpha=\alpha_1,\ldots,\alpha_m$ are the distinct conjugates of $\alpha$ by the elements of $H$, then $\alpha$ is a root of the polynomial
$(x-\alpha_1)\cdots(x-\alpha_m)$, whose roots are distinct and whose coefficients are fixed by $H$.  It follows that $E$ is a separable algebraic extension of the fixed field $E^H$ and any subfield of $E$ containing $E^H$ and generated by finitely many elements and $E^H$ is in fact generated by only one element and $E^H$, and that element satisfies a polynomial of degree at most $|H|$.  But then the degree of $E$ over $E^H$ must be finite and at most $|H|$ (lest some subfield of $E$ have too large a degree over $E^H$), whence the degree of $E$ over $E^H$ must be exactly $|H|$, since any finite extension of a field $F$ admits at most as many $F$-automorphisms as its degree over $F$.  We have shown that {\sl given any field $E$ and a finite group $H$ of automorphisms of it, $E$ is always Galois over the fixed field $E^H$, it has degree $|H|$ over this fixed field, and $H$ is in fact the Galois group of $E$ over $E^H$}.  Returning to our original setting of a finite Galois extension $L$ of a field $K$ with Galois group $G$, we now know that {\sl the map sending a subgroup $H$ of $G$ to the subfield $L^H$ sets up a 1-1 inclusion-reversing correspondence between subfields of $L$ containing $K$ and subgroups of $G$}.  This is the Galois correspondence.  Note that $L$ is always Galois over any intermediate field $L^H$, with Galois group $H$, a subgroup of $G$.

In the Galois correspondence conjugate subgroups $H,xHx^{-1}$ of $G$ correspond to conjugate subfields $L',xL'$.  Hence a subfield $L'$ is preserved by the group $G$ (i.e. its elements are permuted but not necessarily fixed by $G$) if and only if its corresponding subgroup $H$ is normal in $G$; in this case the Galois group of $L'$ over $K$ is the quotient group $G/H$..  Historically the notion of a normal extension of a field preceded that of a normal subgroup of a group; the first person to define the notion of normal subgroup (before the axioms of a group had even been written down) was Galois himself.  

As a simple example of the Galois correspondence, look at the subfield $K=\Bbb Q(\sqrt{2},\sqrt{3})$ of $\Bbb C$ generated by $\Bbb Q$ and $\sqrt{2},\sqrt{3}$.  This extension is Galois; the elements of the Galois group $\Bbb Z_2\times\Bbb Z_2$ each preserve or interchange $\sqrt{2},-\sqrt{2}$ and the same for $\sqrt{3},-\sqrt{3}$.  This group is well known to have three (not two) subgroups of order 2; correspondingly, there are exactly three fields strictly between $\Bbb Q$ and $K$, namely the "obvious" ones $\Bbb Q(\sqrt{2}),\Bbb Q(\sqrt(3)$, and $\Bbb Q(\sqrt{6})$,  It would be tricky (and quite awkward) to show this directly without using Galois theory.