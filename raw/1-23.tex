We conclude our treatment of Galois theory with the Normal Basis Theorem, which generalizes the result of an earlier homework problem to arbitrary finite Galois extensions.  More precisely, let $L$ be finite and Galois over a field $K$, with Galois group $G$.  Then {\sl there is $x\in L$ such that the $G$-conjugates of $x$ form a basis of $L$ over $K$.}  To prove this, we begin by assuming that $K$ is infinite, as we may since if $K$ is finite, then $G$ is cyclic and the result follows from the homework problem mentioned above.  We first derive a criterion for determining when a subset $x_1,\ldots,x_n$ of $L$ is a basis of it over $K$, where $n$ is the degree of $L$ over $K$.  Enumerate the elements of $G$ as $g_1,\ldots,g_n$.  Given $x_1,\ldots,x_n\in L$, form a matrix $M =M(x_1,\ldots,x_n)$ whose $ij$th entry is the image $g_i(x_j)$ of $x_j$ under $g_i$.  {\sl Then the $x_i$ are a basis if and only if the determinant of $M$ is nonzero.}.  Indeed, if there is a nontrivial dependence relation $\sum_i k_i x_i = 0$ among the $x_i$ with $k_i\in K$, then
$\sum_i k_i g_j(x_i) = 0$ for all $j$, so that the columns of $M$ are dependent and its determinant is 0.  Conversely, given a nontrivial dependence relation $\sum_j y_j g_j(x_i) = 0$ among the columns of $M$ with the $y_j$ in $L$, then the $x_i$ cannot span $L$ over $K$, lest this relation amount to a dependence relation $y_j g_j = 0$ among the $g_j$ themselves as $K$-linear transformations of $L$, which we ruled out last quarter (toward the end we showed the elements of $G$ are an $L$-basis for the set of all $K$-linear transformations of $L$).  In particular, the columns (or rows) of our matrix $M$ are dependent over $K$ if and only if they are dependent over $L$.  Next, I claim that {\sl the $g_i$ are algebraically independent over $L$ as $K$ linear transformations of $L$, where we interpret products of the $g_i$ as products, not compositions, of the corresponding functions from $L$ to itself} (we do {\sl not} take the products in $G$).  Indeed, the independence of the $g-i$ over $L$ guarantees that the $n$-tuple $(g_1(x),\ldots,g_n(x)$ runs over an$n$-dimensional $K$-subspace $S$ of $L^n$ whose $L$-span is all of $L^n$, as $x$ runs over $L$; an easy induction using the infiniteness of $L$ shows that such any polynomial vanishing identically on a such a subspace must be 0.  Now we can complete the proof of the Normal Basis Theorem:  set up a matrix $M'$ whose $i$th row is the permutation $g_i g_1,\ldots,g_i g_n$ of $g_1,\ldots,g_n$ (this time we do take products in $G$).  Regarding the $g_i$ as independent variables, we find that the determinant of $M'$ is a nonzero polynomial in the $g_i$ (the coefficient of $g_1^n$ in it is $\pm 1$).  By the algebraic independence of the $g_i$ there is $x\in L$ such that the matrix $M$ obtained from $M'$ by evaluating each of its entries at $x$ is nonzero.  But this matrix $M$ is exactly the one whose nonzero determinant forces $\{g_1(x),\ldots,g_n(x)\}$ to be a $K$-basis of $L$, as desired.  

A beautiful representation-theoretic consequence of this result is that {\sl the field $L$, regarded as module over the group algebra $KG$, is isomorphic to $KG$ itself, the regular representation of $G$ over $K$}.  As a cautionary note, we remark that this does {\sl not} mean that all the structural results that we proved last term about the complex group algebra $\Bbb CG$ carry over to $KG$, as the field $K$ is never algebraically closed in this situation.  For example, suppose $G$ is the cyclic group
$\Bbb Z_3$ and $K$ has a primitive cube root of 1.  Then $G$ has exactly three irreducible representations over $K$ up to equivalence, each of dimension 1 (as it does over $\Bbb C$) and $L$ is isomorphic as a representation of $G$ to the sum of these representations.  On the other hand, if $K$ does not have a primitive cube root of 1, then $G$ has only two irreducible representations over $K$, one of dimension 1, the other of dimension 2, and again $L$ is isomorphic to the sum of these as a representation of $G$.  

We will spend the rest of the course on commutative algebra, starting with a beautiful class of rings that are closely related to the Galois extensions of fields that we have been studying.  We need to recall and generalize a definition that we made last quarter.  Let $L$ be a finite separable but not necessarily Galois extension of a field $K$ and let $L'$ be its normal closure (the splitting field of the product of the minimal polynomials for the elements of say a basis of $L$ over $K$).  If $n$ is the degree of $L$ over $K$, then we know that there are exactly $n$ distinct $K$-homomorphisms f$f_1,\ldots,f_n$ from $L$ into $L'$.  Given $x\in L$, the sum $\sum_i f_i(x)$ of the images of $x$ under these homomorphisms is fixed by the Galois group of $L'$, so must lie in $K$; we call it the {\sl trace Tr$(x)$} of $x$.   If in addition $K=\Bbb Q$ and $x$ happens to be an algebraic integer in $L$, then its trace Tr$(x)$ is a sum of algebraic integers in $\Bbb Q$, so is an integer (as we saw in our treatment of the representation theory of finite groups last quarter).  The other fact we need about the trace in the special case $K=\Bbb Q$ is that {\sl the map sending the ordered pair $(x,y)\n L^2$ to
Tr$(xy)$ is a nondegenerate bilinear form}, that is, that it is blinear (which is clear) and for all $x\in L,x\ne0$ there is $y\in L$ with Tr$(xy)\ne0$; indeed, we need only take $y = x^{-1}$.  In a similar manner, we define the {\sl norm} of any $x\in L$ to be the product of the images $x_i$ of $x$ under the $f_i$; this too lies in the basefield $K$.