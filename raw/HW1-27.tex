\input amstex
\input amssym.def
\loadmsbm
\nopagenumbers
\magnification=\magstep1
\centerline{\bf HW \#3, DUE 1-27}
\bigskip
\centerline{\bf MATH 505A}
\bigskip
\noindent 1. Work out a formula for the number $n(q,p)$ of monic irreducible polynomials of prime degree $p$ over the finite field $F_q$ of order $q$, by arguing that the minimal polynomial of any element of the larger field $F_{q^p}$ that is not in $F_q$ is one such polynomial and that all arise in this way.
\vskip .5in
\noindent 2. Let $p_1,p_2$ be distinct primes.  Extend your reasoning in the previous problem to count the number $n(q,p_1,p_2)$ of monic irreducible polynomials of degree $p_1 p_2$ over $F_q$.
\vskip .5in
\noindent 3. Let $\alpha$ be the positive square root of $(3+\sqrt{3})(2+\sqrt{2})$ in $\Bbb R$ and let $K = \Bbb Q(\alpha)$ be the field generated by $\Bbb Q$ and $\alpha$.  Show that $K$ is also contains $\sqrt{2}$ and $\sqrt{3}$ and is Galois over $\Bbb Q$.
\vskip .5in
\noindent 4. With notation as in the previous problem, show that the Galois group of $K$ over $\Bbb Q$ is isomorphic to the quaternion unit group.
\vskip .5in
\noindent 5. Use Sylow theory and the facts from analysis that every polynomial of odd degree over the reals has a real root and every complex number has a complex square root (you need not prove these assertions) to show that the field $\Bbb C$ of complex numbers is algebraically closed.
\end