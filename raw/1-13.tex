Picking up where we left off last time, we ask when a polynomial $p$ over a field $K$ is {\sl solvable by radicals}; that is, when the splitting field $L$ of $p$ lies inside the last field $K_n$ of a sequence of extensions $K=K_0\subset K_1\subset\cdots\subset K_n$ such that each $K_i = K_{i-1}(\alpha_i)$ with $\alpha_i^{n_i}=\beta_i\in K_{i-1}$ for some positive integer $n_i$.  (We do not assume that $K_i$ is Galois over $K_{i-1}$, nor do we make any a priori assumption about the degrees $n_i$.)  Suppose that there is such a chain of fields $K_i$ for $p$ and let $N$ be the product of the integers $n_i$.  We now extend the field $L$ slightly, letting $K_0'$ be the splitting field of $x^N - 1$ over $K_0$ and inductively letting $K_i'$ be the splitting field of $x^{n_i} - \beta_i$ over $K_{i-1}'$.  Then $K_i'$ is Galois over $K_{i-1}'$ for $i>0$ and its Galois group is a subgroup of $\Bbb Z_{n_i}$, since any $K_{i-1}'$-automorphism of $K_i'$ is determined by what it does to $\alpha_i$, and the only choices are to send it to $\alpha_i$ times some $n_i$th root of 1 in $K_{i-1}'$.  It follows that $K_i'$ is Galois over $K_{i-1}'$ with cyclic Galois group if $i>0$, while $K_0'$ is Galois over $K_0$ with abelian Galois group.  Since a finite abelian group is a direct product of cyclic groups, we may replace the field $K_0'$ by a chain of fields ending in $K_0'$ such that each field is Galois over the previous one with cyclic Galois group.  The upshot is that some extension $L'$ of $L$ is the last field in a chain of fields beginning with $K_0$, with each field a cyclic Galois extension of the previous one.  Applying the Galois correspondence and recalling that the map from subgroups of the Galois group to intermediate fields is inclusion-reversing, we deduce that there is a decreasing chain of subgroups $G_0 > G_1 >\,\ldots\,> G_n = 1$ with $G_0$ the Galois group of $L'$ over $K$, each $G_i$ normal in $G_{i-1]$, and each quotient group $G_{i-1}/G_i$ cyclic.  The existence of such a chain for a fixed group $G_0$ is of course a purely group-theoretic one on $G_0$; amazingly, Galois managed to formulate it and show its equivalence (in his setting) to solvability by radicals *before* the notion of an abstract group had been introduced!  We express this condition by calling the group $G_0$ {\sl solvable}; nowadays most students see it for the first time in a group theory course, but Galois introduced it precisely to characterize the conditions under which a polynomial is solvable by radicals over $\Bbb Q$.  An easy formal consequence of the definition is that any quotient of a solvable group is again solvable; applying this observation to the splitting field $L$ of our original polynomial $p$ over $K$, we see that {\sl if a polynomial over a field is solvable by radicals, then the Galois group of its splitting field must be solvable}.  This is half of the famous Galois Criterion.

Over fields $K$ of characteristic 0, the converse half of the Galois Criterion holds as well:  {\sl a polynomial $p$ over $K$ is solvable by radicals if and only if the Galois group of its splitting field is solvable}.  Before proving this, we take time out to exhibit an example of a quintic polynomial over
$\Bbb Q$ that is not solvable by radicals.  Take $p= x^5 - 4x + 2$.  An easy calculus computation show that $p$ has exactly three real roots (one negative and two positive).  Amazingly enough, this property alone is enough to identify the Galois group $G$ of (the splitting field $S$ of) $p$:  it is the symmetric group $S_5$!  To prove this note first that $S$ is generated over $\Bbb Q$ by the five roots of $p$ in $\Bbb C$, so any element of $G$ permutes these roots and is in turn determined by how it permutes them.  Now $p$ is irreducible over $\Bbb Q$ (by the Eisenstein Criterion), so the degree of $S$ over $\Bbb Q$ must be a multiple of the degree 5 of any of the roots of $p$ over $\Bbb Q$.  It follows that $G$ contains an element of order 5 in $S_5$, which must be a 5-cycle (since 5 is prime).  We also know that complex conjugation is an automorphism of $S$ lying in $G$, which fixed 3 of the roots and flips the other two.  Taking a suitable power of the 5-cycle, we may label the roots
$r_1,\ldots,r_5$ in such a way that two elements of $S_5$ lying in $G$ are the 5-cycle
$(r_1, r_2,\ldots,r_5)$ and the transposition $(r_1,r_2)$.  Conjugating the transposition by the 5-cycle we get that the transpositions $(r_2,r_3),(r_3,r_4),(r_4,r_5)$ all lie in $G$.  But now a standard fact from a first course on group theory is that transpositions of adjacent indices
$(r_1,r_2),\ldots,(r_{n-1},r_n)$ generate all of $S_n$ for any $n$.  We conclude that $S$ has the maximum possible degree of 120 over $\Bbb Q$ and that $G$ is all of $S_5$, as claimed.  Note that it would be essentially impossible to prove this using field theory alone; note also, more generally, that any irreducible polynomial of prime degree $q$ over $\Bbb Q$ with exactly $q-2$ real roots has Galois group $S_q$ over $\Bbb Q$.

Since the only normal subgroups of $S_5$ are 1, the alternating group $A_5$, and $S_5$ itself, we deduce that {\sl no} algebraic expression involving only rational numbers, arithmetic operations, and $n$th roots (for any $n$) can represent a root of $p$}, for if one such expression did represent a root of $p$, than all other roots could be gotten from the same expression by making different choices of roots at some step, so that $p$ would be solvable by radicals over $\Bbb Q$.  It is not difficult (though I will probably not take the time to do it) given any $n\ge5$ (prime or not) to write down a polynomial of degree $n$ over $\Bbb Q$ with Galois group $S_n$:  no such polynomial is solvable by radicals over $\Bbb Q$. 