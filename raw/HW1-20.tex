\input amstex
\input amssym.def
\loadmsbm
\nopagenumbers
\magnification=\magstep1
\centerline{\bf HW \#2, DUE 1-20}
\bigskip
\centerline{\bf MATH 505A}
\bigskip
\noindent 1. Let $K$ be a finite abelian extension of $\Bbb Q$ (i.e. a finite Galois extension with abelian Galois group), regarded as a subfield of the complex numbers.  Let $\alpha\in K$ be an algebraic integer whose complex norm is 1.  Show that $\alpha$ is a root of 1, by first showing that all Galois conjugates of $\alpha$ also have norm 1, as do all Galois conjugates of all powers of $\alpha$, and then arguing that all powers of $\alpha$ are roots of some monic polynomial with bounded degree and bounded integral coefficients; thus all such powers are roots of one of finitely many polynomials $p_1,\ldots,p_m$ over $\Bbb Z$.  Deduce that all entries in the character table of a finite group with complex norm 1 are roots of 1.
\vskip .5in
\noindent 2. Let $L$ be a finite cyclic extension (Galois with a cyclic Galois group $G$) of a field $K$.  Show that there is $\alpha\in L$ such that the $G$-conjugates of $\alpha$ form a $K$-basis of $L$.  (Let $g$ be a generator of $G$, say with order $n$.  Then $g$ is in particular a $K$-linear transformation from $L$ to itself of order $n$.  Use the invariant factor decomposition of such a transformation to write $L$ as a direct sum of quotients $K[x]/(p_1),\ldots,K[x]/(p_m)$ as a $K[x]$-module, where $p_1 | p_2 |\cdots | p_m$; finally use the linear independence of the elements of $G$ as linear transformations of $L$ to show that $m=1$ and $p_m = x^n - 1$.)
\vskip .5in
\noindent 3. Show that the polynomial $x^{p^n} - x - 1$ is irreducible over the field $\Bbb Z_p$ for $p$ prime if and only if either $n=1$ or $n = p = 2$.  (This polynomial is irreducible if and only if the Galois group of its splitting field acts transitively on its roots; a generator for this Galois group is the Frobenius automorphism of its splitting field.)
\vskip .5in
\noindent 4. Let $\alpha$ be a root of an irreducible polynomial of degree 4 over $\Bbb Z_3$.  Determine the other roots of this polynomial in terms of $\alpha$; the answer does not depend on the choice of polynomial.
\vskip .5in
\noindent 5. Show that the polynomial $x^4 + 1$ is irreducible over $\Bbb Z$ or $\Bbb Q$ but reducible over $\Bbb Z_p$ for any prime $p$, by looking at elements of order 8 in a suitable extension of $\Bbb Z_p$.
\end